% !TeX root = ./thesis.tex
% !TeX spellcheck = hu_HU
% !TeX encoding = UTF-8
% !TeX program = pdflatex
% !BIB program = bibtex
%TODO Change language to en_GB (recommended) or en_US for English documents
\documentclass[12pt,a4paper,oneside]{report}             % Egyoldalas (javasolt)
%\documentclass[11pt,a4paper,twoside,openright]{report}  % Duplex

\input{include/packages}

\makeatletter
    \setlength\@fptop{0\p@}
\makeatother


%TODO Saját adataiddal töltsd ki a kommentek szerint
%--------------------------------------------------------------------------------------
\newcommand{\szerzoVezeteknev}{Lengyel}
\newcommand{\szerzoKeresztnev}{Márk}
\newcommand{\szerzoNeptun}{LNXQYO}

\newcommand{\szakirany}{\merninf{}} % automat vagy infokom

\newcommand{\konzulensAMegszolitas}{}
\newcommand{\konzulensAVezeteknev}{Hollósi}
\newcommand{\konzulensAKeresztnev}{János}
\newcommand{\konzulensBMegszolitas}{}
\newcommand{\konzulensBVezeteknev}{Gerendás}
\newcommand{\konzulensBKeresztnev}{Péter}
\newcommand{\konzulensCMegszolitas}{}
\newcommand{\konzulensCVezeteknev}{}
\newcommand{\konzulensCKeresztnev}{}

\newcommand{\cim}{Snooker billiárdjáték elemzése} % Cím
\newcommand{\tanszek}{\szeint} % automatizálási (\szeaut) vagy távközlési (\szetat)
\newcommand{\doktipus}{\szakdolgozat} % Dokumentum típusa (\szakdolgozat, \diplomaterv vagy \dolgozat)
\newcommand{\szak}{\minfBSc} % villamosmérnöki msc (\villMSc) vagy villamosmérnöki bsc (\villBSc)

%TODO Nyelv beállítása
% Beállítások magyar nyelvű dolgozathoz
\input{include/thesis-hu}
% Settings for English documents
%\input{include/thesis-en}


\newcommand{\szerzoMeta}{\szerzoVezeteknev{} \szerzoKeresztnev} % egy szerző esetén TODO@FMA két szerző
\input{include/preamble} % beállítások, nem kell vele foglalkoznod remélhetőleg, de ha valami latex hekkelésre vagy új parancsra van szükséged annak itt a helye


%--------------------------------------------------------------------------------------
% Itt kezdődik a dolgozat
%--------------------------------------------------------------------------------------
%\setstretch{1.5} % sorköz beállítása
%\spacing{1.5}
%\linespread{1.25}
\onehalfspacing

\newenvironment{codewrapper}{
    \vspace{5mm}
    \hspace{-11mm}
    \begin{minipage}{\linewidth}
}{
    \end{minipage}
}

\makeatletter %setting inline code style
\lstdefinestyle{mystyle}{
  basicstyle=%
    \ttfamily
    \lst@ifdisplaystyle\scriptsize\fi
}
\makeatother
\lstset{style=mystyle}

\begin{document}

%TODO Feladatkiíró lap helye, csak a nyomtatott verzijóba kerül az eredeti példány
%~~~~~~~~~~~~~~~~~~~~~~~~~~~~~~~~~~~~~~~~~~~~~~~~~~~~~~~~~~~~~~~~~~~~~~~~~~~~~~~~~~~~~~
%\include{include/project}


\selectthesislanguage


% Címoldal
%~~~~~~~~~~~~~~~~~~~~~~~~~~~~~~~~~~~~~~~~~~~~~~~~~~~~~~~~~~~~~~~~~~~~~~~~~~~~~~~~~~~~~~
\hypersetup{pageanchor=false}
%--------------------------------------------------------------------------------------
%	The title page
%--------------------------------------------------------------------------------------
\begin{titlepage}
\begin{center}

\begin{flushleft}
	\hspace{-2cm}\includegraphics[width=70mm,keepaspectratio]{figures/infologo_2020_university.png} \\
	\hspace{-2cm}\includegraphics[width=70mm,keepaspectratio]{figures/infologo_2020_department.png}
\end{flushleft}

\vspace{120pt} %because it's the top
{\Huge \bfseries \MakeUppercase {\doktipus}}\\
\vspace{68pt}
{\huge \bfseries \cim}\\
\vspace{68pt}
{\huge \bfseries{\szerzo}}

\vspace{90pt}
\Large \textbf{\szak{}}\\
\textbf{\szakirany}\\
\vspace{90pt}
{\Large \textbf{\the\year}}

\vfill

\end{center}
\end{titlepage}
\hypersetup{pageanchor=false}



% Témaválasztó, Nyilatkozat és Kivonat
%~~~~~~~~~~~~~~~~~~~~~~~~~~~~~~~~~~~~~~~~~~~~~~~~~~~~~~~~~~~~~~~~~~~~~~~~~~~~~~~~~~~~~~
\includepdf{include/temavalasztas2.pdf} % témaválasztó adatlap
\include{include/declaration} % ez legenerálódik magától a fentebb megadott adatok alapján
\pagenumbering{gobble} % roman numbering as default
\setcounter{page}{1}

\selecthungarian

%----------------------------------------------------------------------------
% Kivonat Magyarul 
%----------------------------------------------------------------------------
\chapter*{Kivonat}%\addcontentsline{toc}{chapter}{Kivonat}

A dolgozat célja egy snooker játék felülnézetes videójának felismerése, azon található golyók helyzetének és színének megállapítása és különböző szempontok vizsgálata a golyók helyzetének változása alapján. A dolgozat nem csak megoldást nyújt egy snooker asztal felismeréséhez, hanem részletezi a megvalósításhoz használt képfeldolgozási és gépi tanulási eszközöket és ezen eszközökhöz használt TensorFlow és OpenCV könyvtárak metódusait és azok használatát.
\par A dolgozat tartalma képfeldolgozás terén legfőképp a HSV maszkolás, kontúrok vizsgálata és kördetektálás témákat, míg gépi tanulási módszerek terén a konvolúciós neurális hálózatokat és azok betanításához szükséges adatkészlet feldolgozását érinti.


\vfill
\selectenglish


%----------------------------------------------------------------------------
% Abstract in English
%----------------------------------------------------------------------------
\chapter*{Abstract}%\addcontentsline{toc}{chapter}{Abstract}

The aim of this thesis is to identify the position of the balls in a top view video of a snooker game, to determine the position and color of the balls and to investigate different aspects of the game based on the change in the position of the snooker balls. The thesis not only provides a solution for the recognition of a snooker table, but also introduces image processing and machine learning tools used for the implementation of the application, and the methods and usage of the TensorFlow and OpenCV libraries used to implement these tools.
\par The content of the thesis is mainly related to HSV masking, contour analysis and circle detection in the field of image processing, while in the field of machine learning it is related to convolutional neural networks and the processing of the dataset required for their training.

\vfill
\selectthesislanguage

\newcounter{romanPage}
\setcounter{romanPage}{\value{page}}
\stepcounter{romanPage} %TODO ezt át kell írnod


% Tartalomjegyzék
%~~~~~~~~~~~~~~~~~~~~~~~~~~~~~~~~~~~~~~~~~~~~~~~~~~~~~~~~~~~~~~~~~~~~~~~~~~~~~~~~~~~~~~
\pagenumbering{gobble}
\tableofcontents\vfill
\addtocontents{toc}{\protect\thispagestyle{empty}}


% Ábrák listája - a word-ös sablon szerint szükséges
%~~~~~~~~~~~~~~~~~~~~~~~~~~~~~~~~~~~~~~~~~~~~~~~~~~~~~~~~~~~~~~~~~~~~~~~~~~~~~~~~~~~~~~
\clearpage\phantomsection
\listoffigures
\addcontentsline{toc}{chapter}{\listfigurename}


% A dolgozat lényegi része
%~~~~~~~~~~~~~~~~~~~~~~~~~~~~~~~~~~~~~~~~~~~~~~~~~~~~~~~~~~~~~~~~~~~~~~~~~~~~~~~~~~~~~~
\pagenumbering{arabic}

%TODO készítsd el a saját munkád
\chapter{\bevezetes}

\section{A projekt célja}
A dokumentumban szereplő projekt célja snooker billiárdjáték felismerése, és analizálása fénykép/képernyőfelvétel alapján. A felismerés az asztalon elhelyezkedő különböző színű golyók pozíciójának meghatározásából áll. A felismerés megvalósításához különféle képfeldolgozási eszközöket, neurális hálózat alapú kép osztályozást használok, amelyeket \textbf{Python} programozási nyelven valósítok meg főként \textbf{OpenCV} és \textbf{Tensorflow} könyvtárak használatával. A bevezetés későbbi részeiben ismertetem a snooker billiárdjátékot, továbbá a projekt elkészítéséhez használt programozási nyelvet és fejlesztési könyvtárakat.

\section{A Snooker játék}
\subsection{Általánosságban a snooker játékról}
A snooker a billiárdjátékok egy bizonyos fajtája, amelyet egy zöld színű posztóval bevont asztalon játszanak, amelynek mérete általában 12 x 6 láb (365,8 cm x 182,9 cm)\cite{snooker_rules}. Az asztal négy sarkában és a két hosszabb oldal felénél ún. \textbf{zsebek} helyezkednek el. A játék célja a színes golyók belökése a fehér golyó segítségével a fent említett zsebekbe.

\subsection{Eszközök}
A játékot két fél játssza egymás ellen. A felek a lökéseiket egy hosszúkás, fából készült eszköz segítségével végzik. Ezt az eszközt \textbf{dákónak} nevezzük. A dákó vége, amellyel a golyó elütésre kerül, bőrrel van bevonva, amely a golyóval való kapcsolatot javítja. A dákón kívül a golyó elütéséhez a játékosok használhatnak segédeszközöket.
\par A tartozékok részei továbbá a már eddig is szóba került golyók. A játékhoz \textbf{22 db színes golyó} tartozik amelyek átmérője 52,5 mm.\cite{snooker_rules}
\newline Az egyes golyók különböző pontértékekkel rendelkeznek:\cite{snooker_rules}
\begin{itemize}
    \setlength\itemsep{-2pt}
    \item 1 db Fehér
    \item 15 db Piros - 1 pont
    \item 1 db Sárga - 2 pont
    \item 1 db Zöld - 3 pont
    \item 1 db Barna - 4 pont
    \item 1 db Kék - 5 pont
    \item 1 db Rózsaszín - 6 pont
    \item 1 db Fekete - 7 pont
\end{itemize}
A fehér golyó nem rendelkezik pontértékkel, mivel a játékosok ezt a golyót használják lökéseikhez.
\par A játék egy menetét \textbf{frémnek} nevezik, amely a kezdő lökéstől a fekete golyó elhelyezéséig tart.\cite{snooker_rules}
\begin{figure}[!ht]
    \centering
    \includegraphics[width=100mm, keepaspectratio]{figures/starting_position.png}
    \caption{A golyók kezdeti pozíciója.}
    \label{fig:kezdeti_pozicio}
\end{figure}

\subsection{Pontszerzés}
A játékosok a pontjaikat a golyók bizonyos sorrendben való zsebbe helyezésével szerzik. Az egymás után hiba nélkül szerzett pontok összegét \textbf{törésnek} nevezzük. Egy játékos például szerzhet 9 pontos törést a következő golyók egymás utáni elhelyezésével \textit{piros -> zöld -> piros -> barna}.\cite{shamos2002new}
\par Egy játékos büntetőpontokat kap hibák elkövetése esetén. Hibát elkövetni lehet például a fehér golyó zsebbe helyezésével, nem megfelelő színű golyó elütésével. Az elkövetett hibáért minimum 4, maximum 7 pontlevonás jár, attól függően, hogy milyen színű golyók mozdulnak a hiba elkövetésekor (pl.: Ha a cél a piros golyó lelökése, de a lövő a feketét találja el, akkor 7 hibapont jár). A hibát elkövető játékos a törésének pontjait megkapja a hibát elkövetett lövés közben elhelyezett golyók pontjainak kivételével.\cite{snooker_rules}

\section{Az OpenCV képfeldolgozási könyvtár}
Az OpenCV egy főként \textbf{valós idejű képfeldolgozáshoz} használt programozási függvénykönyvtár. A könyvtár többféle programozási nyelvekhez készült implementációval létezik (pl.: C++, Python, Java stb.)\cite{opencv_library}, ezek közül ebben a projektben Python programozási nyelven keresztül fogom használni.
\par A könyvtárból használt függvények segítségével kerülnek megnyitásra a képek, továbbá a képeken való műveletek (pl.: szürkeárnyalatolás, élkeresés) is a könyvtár segítségével lesznek végrehajtva. A későbbiekben lesz szó a könyvtárból használt függvényekről, azok működéséről nagyobb részletességben.

\section{Tensorflow a neurális hálózatokhoz}
A Tensorflow az OpenCV -hez hasonlóan egy függvénykönyvtár, azzal a különbséggel, hogy a könyvtár a \textbf{neurális hálózatok elkészítését és betanítását} teszi lehetővé.\cite{tensorflow} A neurális hálózatok közül itt főként neurális hálózatokat (Neural Network) fogok használni, amelyek a képfeldolgozás, kép osztályozás területén teljesítenek kiemelkedően. A könyvtár eszközeiről szintén részletesebben beszélek majd a későbbi fejezetekben.
\chapter{A Snooker játék}

\section{Általánosságban a snooker játékról}
\label{section:snooker_altalanos}
A snooker a billiárdjátékok egy bizonyos fajtája, amelyet egy zöld színű posztóval bevont asztalon játszanak, amelynek mérete általában 12 x 6 láb (365,8 cm x 182,9 cm)\cite{snooker_rules}. Az asztal négy sarkában és a két hosszabb oldal felénél ún. \textbf{zsebek} helyezkednek el. A játék célja a színes golyók belökése a fehér golyó segítségével a fent említett zsebekbe.

\section{Eszközök}
A játékot két fél játssza egymás ellen. A felek a lökéseiket egy hosszúkás, fából készült eszköz segítségével végzik. Ezt az eszközt \textbf{dákónak} nevezzük. A dákó vége, amellyel a golyó elütésre kerül, bőrrel van bevonva, amely a golyóval való kapcsolatot javítja. A dákón kívül a golyó elütéséhez a játékosok használhatnak segédeszközöket.
\par A tartozékok részei továbbá a már eddig is szóba került golyók. A játékhoz \textbf{22 db színes golyó} tartozik amelyek átmérője 52,5 mm.\cite{snooker_rules}
\newline Az egyes golyók különböző pontértékekkel rendelkeznek:\cite{snooker_rules}
\begin{itemize}
    \setlength\itemsep{-2pt}
    \item 1 db fehér,
    \item 15 db piros (1 pont),
    \item 1 db sárga (2 pont),
    \item 1 db zöld (3 pont),
    \item 1 db barna (4 pont),
    \item 1 db kék (5 pont),
    \item 1 db rózsaszín (6 pont),
    \item 1 db fekete (7 pont).
\end{itemize}
A fehér golyó nem rendelkezik pontértékkel, mivel a játékosok ezt a golyót használják lökéseikhez.
\par A játék egy menetét \textbf{frémnek} nevezik, amely a kezdő lökéstől a fekete golyó elhelyezéséig tart.\cite{snooker_rules}
\begin{figure}[!ht]
    \centering
    \includegraphics[width=100mm, keepaspectratio]{figures/starting_position.png}
    \caption{A golyók kezdeti pozíciója.}
    \label{fig:kezdeti_pozicio}
\end{figure}

\section{Pontszerzés}
A játékosok a pontjaikat a golyók bizonyos sorrendben való zsebbe helyezésével szerzik. Az egymás után hiba nélkül szerzett pontok összegét \textbf{törésnek} nevezzük. Egy játékos például szerzhet 9 pontos törést a következő golyók egymás utáni elhelyezésével \textit{piros -> zöld -> piros -> barna}.\cite{shamos2002new}
\par Egy játékos büntetőpontokat kap hibák elkövetése esetén. Hibát elkövetni lehet például a fehér golyó zsebbe helyezésével, nem megfelelő színű golyó elütésével. Az elkövetett hibáért minimum 4, maximum 7 pontlevonás jár, attól függően, hogy milyen színű golyók mozdulnak a hiba elkövetésekor (pl.: Ha a cél a piros golyó lelökése, de a lövő a feketét találja el, akkor 7 hibapont jár). A hibát elkövető játékos a törésének pontjait megkapja a hibát elkövetett lövés közben elhelyezett golyók pontjainak kivételével.\cite{snooker_rules}
\chapter{Felhasznált szoftverek}

\section{Az OpenCV képfeldolgozási könyvtár}
Az OpenCV (Open Source Computer Vision Library) egy főként \textbf{valós idejű képfeldolgozáshoz} használt programozási függvénykönyvtár. A könyvtár többféle programozási nyelvekhez készült implementációval létezik (pl.: C++, Python, Java stb.)\cite{opencv_library}, amelyek közül ebben a projektben a Python programozási nyelvhez készült verziót fogom használni. A szoftver szabadon használható az \textit{Apache License 2.0} alatt.
\par A függvénykönyvtár fejlesztését az Intel Research kezdeményezte 1999-ben, a CPU intenzív alkalmazások fejlődése érdekében. A projekt lefőbb hozzájárulói az Intel Russia optimalizálással foglalkozó szakemberei, továbbá az Intel Performance Library csapata. \cite{kaehler2016learning}
\par Kezdetben az OpenCV létrehozásának célja volt, hogy \textbf{nyílt}, \textbf{optimalizált} kódot képezzen gépi látáshoz, valamint, hogy egy \textbf{egységes infrastruktúrát} biztosítson a fejlesztőknek a területen, ezzel megkönnyítve a programkódok olvashatóságát és terjeszthetőségét. Cél volt még, hogy fejlesszék a gépi látásra alapuló kereskedelmi felhasználást, hordozható, teljesítményorientált programkód létrehozásával.\cite{bradski2008learning}
\par Az OpenCV-t sokféle területen használják, ezek közül néhány:
\begin{itemize}
    \setlength\itemsep{-2pt}
    \item arcfelismerő rendszerek,
    \item gesztusok felismerése,
    \item objektumok felismerése,
    \item szegmentálás és felismerés,
    \item mozgás felismerés,
    \item kiterjesztett valóság.
\end{itemize}

\par A dolgozat keretében a könyvtárból használt függvények segítségével kerülnek megnyitásra a képek, továbbá a képeken való műveletek (pl.: szürkeárnyalatolás, élkeresés) is a könyvtár segítségével lesznek végrehajtva. A későbbiekben lesz szó a könyvtárból használt függvényekről, azok működéséről nagyobb részletességben.

\section{Tensorflow gépi tanulási könyvtár}
A Tensorflow az OpenCV -hez hasonlóan egy függvénykönyvtár, amely a \textbf{neurális hálózatok elkészítését és betanítását} teszi lehetővé. A Tensorflow egy nyílt forráskódú szoftver könyvtár, használható különböző feladatok elvégzésére is, de főként mély neurális hálózatok betanítására és azokkal való következtetésekre, becslésekre használható. \cite{tensorflow2015-whitepaper}
\par A Tensorflow a Google Brain csapat által lett kifejlesztve a Google saját kutatásaihoz. Az első verzió az \textit{Apache License 2.0} szoftverlicensz alatt jelent meg, majd később 2019 szeptemberében megjelent a Tensorflow frissített verziója, amelyet Tensorflow 2.0 nak neveztek el.\cite{tensorflow2015-whitepaper}
\par A könyvtár használható különböző programozási nyelvekben is (Python, C++, Javascript, stb.) amelynek köszönhetően flexibilisen használható külonféle alkalmazásokban. A Tensorflow sokféle funkcióval rendelkezik, ezek közül a következőkben néhányat részletesen megemlítek.
\par A Tensorflow funkciói közé tartozik, hogy támogatja az \textbf{automatikus differenciálás} (Automatic differentiation) folyamatát, amellyel automatikusan kiszámolható a gradiens vektor egy modelhez, annak paramétereit figyelembe véve. Ez a folyamat különösen hasznos a visszaterjesztéses (Backpropagation) modelleknél, ahol gradiensekre van szükség az optimalizáláshoz. Ahhoz, hogy ez megvalósítható legyen a keretrendszer számon tartja a modell bemenetére végzett műveleteket, majd a modell paramétereitől függően kiszámolja a gradienseket. \cite{tensorflow2015-whitepaper,tensorflow_autodiff}
\par A könyvtár lehetővé teszi továbbá, a számítások \textbf{elosztását} különböző hardver eszközök közt, ezzel a tanítás és kiértékelés folyamatok nagymértékben felgyorsíthatóak főként komplex, több rétegű modellek esetén.\cite{tensorflow2015-whitepaper}
\par A modellek tanításához nélkülözhetetlen a \textbf{költségfüggvények használata}, ezek rendelkezésre állnak a könyvtárban. A költségfüggvények szerepe, hogy kiszámolják a \textit{"hibát"}, vagy másnéven eltérést a modell kimenete, és annak az adott bemenethez tartozó elvárt kimenete közt, amely érték segítségével a modell hangolni tudja paramétereit.
\par A Tensorflow könyvtárban megtalálható egy ún. \textbf{\lstinline{TF.nn}} modul, amelynek segítségével egyszerű műveletek végezhetőek neurális hálózatokon. Ilyen műveletek lehetnek konvolúciók képfelismeréshez, aktivációs függvények (Softmax, RELU, Sigmoid, stb.) és egyéb primitív műveletek.
\par A könyvtár részét képezik különböző \textbf{optimalizálók}, amelyek a modellek betanításában játszanak szerepet, különböző optimalizálók lehetővé teszik a paraméterek különféle módon való hangolását, ezzel kihatva a modell teljesítményére. Az ilyen optimalizálók közül az egyik legismertebb az \textit{ADAM Optimizer}, amelyet ebben a munkában is felhasználok.
\par Neurális hálózatok közül főként \textbf{konvolúciós neurális hálózatokat} (Convolutional Neural Network) fogok használni a későbbiekben, amelyek a képfeldolgozás, kép osztályozás területén teljesítenek kiemelkedően. A könyvtár egyes eszközeiről szintén részletesebben beszélek majd a megvalósítással kapcsolatos fejezetben.
\chapter{A program megtervezése}
\label{chapter:program_tervezes}
\section{Az asztal felismerése}
\label{section:asztal_felismeres}

Ebben az alfejezetben a program első folyamatáról lesz szó, ami nem más mint az asztal, vagy más néven játékterület felismerése. A program ezen része bemenetként fog fogadni egy képet. Ez a kép származhat \textbf{videófelvételből} (videófelvétel egy képkockája), \textbf{valós idejű felvételből} (szintén egy képkocka), vagy szimplán egy \textbf{képfájlból}. A program bemutatásához egy felülnézetből készített snooker játszma felvételét fogom használni.
\par Az asztalfelismerő lépés kicsit elkülönül a többi lépéstől, hiszen a további lépések különféle bemenetszerzési módszerektől függetlenül, megegyezően hajtódnak végre. Ahhoz hogy a további lépések pontosak legyenek és megfeleő teljesítménnyel működjenek, az asztalt mindig azonos méretben, elforgatásban és torzításban kell megkapniuk.
\par Ebben a módszerben az elforgatást leszámítva a detektálás, átméretezés és a torzítás lesz középpontban. Az elforgatást nem veszem figyelembe, hiszen a bemenetről feltételezhető, hogy bizonyos orientációban áll rendelkezésre.

\par A valódi bemeneti kép, függetlenül annak előállítási módszerétől, a \ref{fig:bemeneti_kep} képen látható.

\begin{figure}[!ht]
    \centering
    \includegraphics[width=110mm, keepaspectratio]{figures/input_screen.png}
    \caption{Egy nyers bemeneti kép a felvételről.}
    \label{fig:bemeneti_kep}
\end{figure}

\par Ezen a képen kell megtalálni a játékterület zöld részét. Ez könnyen megtehető a kép ún. HSV (Hue Saturation Value) formátumra való átalakításával. Ennek a konverziónak a kimenete a \ref{fig:bemeneti_kep_hsv} képen látható.

\begin{figure}[!ht]
    \centering
    \includegraphics[width=110mm, keepaspectratio]{figures/input_screen_hsv.png}
    \caption{A \ref{fig:bemeneti_kep} kép HSV re alakított verziója RGB reprezentációban.}
    \label{fig:bemeneti_kep_hsv}
\end{figure}

\par Ez az átalakítás azért fontos, mert HSV formátumban könnyebben intervallumok közé lehet szorítani a játékterület zöld színét. A HSV konverzió belső működéséről részletesebben a megvalósítás részben lesz szó.
\newline A HSV konverzió során kapott intervallumok:
\begin{itemize}
    \setlength\itemsep{-2pt}
    \item árnyalat (Hue),
    \item telítettség (Saturation),
    \item érték (Value).
\end{itemize}

\par A HSV konverzió után az adott specifikus intervallumon kívül helyezkedő értékek maszkolásra kerülnek, hátrahagyva a kívánt játékterület képpontjait. A kép a maszkolás után visszaalakításra kerül az eredeti RGB formátumára. A folyamat után kapott kép a \ref{fig:bemeneti_kep_mask} képen látható.

\begin{figure}[!ht]
    \centering
    \includegraphics[width=110mm, keepaspectratio]{figures/input_screen_mask.png}
    \caption{A \ref{fig:bemeneti_kep} kép a maszkolás után.}
    \label{fig:bemeneti_kep_mask}
\end{figure}

\par Az előző folyamat után már jól látható a játékterület, azonban vannak apró foltok amik nem kerültek maszkolásra. Ezek a foltok hasonló HSV értékekkel rendelkeznek, mint a játékterület, kiszűrésük megoldható a kontúrok megkeresésével, majd feltételezve, hogy a legnagyobb folt a maszkolt képen a játékterület, annak kiválasztásával. Ezzel az eljárással már megadható a játékterület kontúrja. Szintén feltételezve, hogy ez négy sarokpontból áll, és egy téglalap pontosan határolja, a játékterület képe \ref{fig:bemeneti_asztal2} megkapható a kontúr eredeti képből való kivágásával és átméretezésével.

\begin{figure}[!ht]
    \centering
    \includegraphics[width=110mm, keepaspectratio]{figures/input_table.png}
    \caption{Az eredeti képből kinyert játékterület.}
    \label{fig:bemeneti_asztal2}
\end{figure}

\par Fent említettem, hogy feltételezhető, hogy a játékterület kontúrja a kontúrkeresés után téglalap alakú, és a kontúrt alkotó pontok száma 4. Viszont valóságban ez nehezen fordul elő, ezért szükséges a pontok számának leszűkítése és a négy pontot határoló alakzatban lévő kép torzítása 2:1 oldalarányú téglalapra. Az oldalarány a szabvány snookerasztal 12 x 6 láb (365,8 cm x 182,9 cm)\cite{snooker_rules} méretéből következik. A pontok szűkítéséről és a torzításról a megvalósítás fejezetben lesz részletesetbben szó.

\section{A golyók azonosítása}
A golyók azonosítását különféle mószerekkel lehet elvégezni, ezek eltérnek sebességben és pontosságban. A folyamatok bemenete az előzőekben megismert asztal felismerés kimenete lesz, kimenetük pedig a \ref{tab:felismert_koordinatak} táblázatban látható x és y pozíciók, adott golyók színe szerint. A folyamat belső működése módszerenként eltér egymástól, ezeket a módszereket az implementáció egyes iterációjának részeként használtam, majd változtattam meg az elért teljesítmény növeléséhez.

\begin{table}[!ht]
    \caption{A golyó felismerés kimeneti adatai a \ref{fig:bemeneti_asztal2} kép alapján.}
    \label{tab:felismert_koordinatak}
	\footnotesize
	\centering
	\begin{tabular}{ l l l }
		\toprule
		golyó színe & x pozíció [0, 1023] & y pozíció [0, 511] \\
		\midrule
        fehér     & 298.5 & 283.5\\
        piros 0   & 244.5 & 204.5\\
        piros 1   & 242.5 & 243.5\\
        piros 2   & 323.5 & 159.5\\
        piros 3   & 190.5 & 260.5\\
        piros 4   & 202.5 & 222.5\\
        piros 5   & 216.5 & 259.5\\
        piros 6   & 268.5 & 227.5\\
        piros 7   & 144.5 & 192.5\\
        piros 8   & 625.5 & 451.5\\
        piros 9   & 166.5 & 234.5\\
        piros 10  & 202.5 & 247.5\\
        piros 11  & 231.5 & 254.5\\
        piros 12  & 223.5 & 234.5\\
        sárga     & 797.5 & 174.5\\
        zöld      & 797.5 & 331.5\\
        barna     & 536.5 & 291.5\\
        kék       & 512.5 & 252.5\\
        rózsaszín & 285.5 & 253.5\\
        fekete    & 120.5 & 211.5\\
		\bottomrule
	\end{tabular}
\end{table}

\subsection{Azonosítás mintaillesztéssel}
Ennek a módszernek az alapja tisztán mintaillesztéssel működik. A mintaillesztés egy arányos méretű képet illeszt rá a játékterület képére, majd amennyiben az illeszkedés mértéke meghalad egy bizonyos küszöbértéket, a mintaillesztés iterációjának a pozíciója mentésre kerül. Ebből a pozícióból meghatározható a golyó helyzete. A mintaillesztéshez használt minta a \ref{fig:minta_kep} ábrán látható.

\begin{figure}[!ht]
    \centering
    \includegraphics[width=30mm, keepaspectratio]{figures/template_red.png}
    \caption{A mintaillesztéshez használt minta.}
    \label{fig:minta_kep}
\end{figure}

\par A \ref{fig:minta_kep} minta felbontása láthatóan alacsony, azonban túl nagy felbontás esetén a folyamat meglehetősen lassabban megy végbe, továbbá a minta méretét még a felismert játékterület mérete is meghatározza.
\par A mintaillesztés problémái közé tartozik, hogy a zöld golyó mintaillesztésénél a küszöbértéket beállítani nehéz, és az eredmény pontatlan, lásd \ref{fig:rossz_zold}.

\begin{figure}[!ht]
    \centering
    \includegraphics[width=110mm, keepaspectratio]{figures/wrong_green.png}\hspace{2mm}
	\includegraphics[width=110mm, keepaspectratio]{figures/green_ok.png}
    \caption{A zöld golyó mintaillesztésének hibája (felül) és annak orvoslása HSV konverzióval (alul).}
    \label{fig:rossz_zold}
\end{figure}

\par A \ref{fig:rossz_zold} ábrán látható hiba valamelyest orvosolható a kép és minta HSV -re való konvertálásával. Ez a konverzió jó eredményeket ad, azonban nagyon minimálisan csökkenti a teljesítményt. A mintaillesztés sajnos prolémákba ütközik a piros és barna golyók megkülönböztetésekor is. Ez a \ref{fig:rossz_barna} képen látható.

\begin{figure}[!ht]
    \centering
    \includegraphics[width=110mm, keepaspectratio]{figures/wrong_brown.png}\hspace{2mm}
	\includegraphics[width=110mm, keepaspectratio]{figures/brown_ok.png}
    \caption{A barna golyó mintaillesztésének hibája (felül) és annak orvoslása a piros golyók levonásával (alul).}
    \label{fig:rossz_barna}
\end{figure}

\par Itt a színek közelsége miatt nehéz megkülönböztetni a golyókat, ezért a piros és barna golyók nagyon minimálisan térnek el a \ref{for:cross_correlation} függvény miatt. Ez a probléma HSV konvertálás után is fennáll. Erre a megoldás, hogy az érzékelt piros golyók kivonásra kerülnek a barna golyók listájából. Ahhoz, hogy pontos legyen az eredmény, viszont szükséges, hogy a piros golyók megfelelően legyenek érzékelve, amely nem minden esetben biztosítható, ezért ez a módszer nem túl pontos bizonyos bemenetekre. Hasonló problémák merülhetnek fel a piros és rózsaszín, továbbá a fehér és rózsaszín golyók felismerésekor is.
\par Annak ellenére, hogy a módszer nem túl optimális, jól használható adatkészeletek készítésére, hiszen a felsmerést nagyrészt helyesen megoldja és a problémák ismeretében a kézzel válogatást nagymértékben megkönnyíti.

\subsection{Azonosítás körkeresés és mintaillesztéssel}
Az előző módszerhez hasonlóan a golyó színek szerinti osztályozása itt is mintaillesztéssel történik, azonban a sebesség növelése érdekében először a golyók helyzetét egy kördetektáló algoritmus határozza meg, majd a kapott értékek jutnak tovább a mintaillesztő algoritmushoz. Ez az egész képen való pásztázás és mintaillesztéshez képest a teljesítményt a minták leszűkített mennyiségének köszönhetően nagymértékben növeli.

\par A körök megtalálásához a körkereső algoritmusnak meg kell adni néhány paramétert, ezek közé tartoznak:
\begin{itemize}
    \setlength\itemsep{-2pt}
    \item a keresett körök minimális és maximális sugara,
    \item a keresett körök közti minimális távolság, duplikációk szűréséhez,
    \item az ellenőrzött alakzatok körrel való hasonlóságának küszöbértéke.
\end{itemize}
\par Az algoritmus lefutás után megadja a bemeneti játékterületen talált kör alakú kontúrokat, ezeket a \ref{fig:talalt_korok} ábra szemlélteti. A körkereső algoritmusról a megvalósítás fejezetben írok részletesebben.

\begin{figure}[!ht]
    \centering
    \includegraphics[width=110mm, keepaspectratio]{figures/detected_circles.png}
    \caption{A Hough transzformáció lefutása után kapott körök.}
    \label{fig:talalt_korok}
\end{figure}

\par Ez a módszer körök megtalálására jól használható, a mintaillesztéshez szükséges képek könnyedén kivághatók az eredeti képből a körök paraméterei alapján. A probléma szintén a mintaillesztéssel van, hiszen a kivágott körök az előzőleg megismert mintaillesztéssel kerülnek beazonosításra. Ez sajnos az eddig megismert hibákat vonja maga után, annak ellenére, hogy a sebesség javul. Viszont akárcsak a szimpla mintaillesztéses módszer ez a módszer is alkalmas adatkészletek elkészítésére, és a kapott adatok kézzel ellenőrzését nagymértékben megkönnyíti.

\subsection{Azonosítás körkeresés és gépi tanulás segítségével}
A mintaillesztés hibáinak kiküszöböléséhez, az osztályozás elvégezhető neurális hálózat segítségével. Ez a játékterület egy kerettel való képpontonkénti végigpásztázásával szintén megoldható, azonban körkeresésnél megismert Hough transzformációval jobb teljesítmény érhető el.

\par A következőkben a körkeresés eredményeképp kapott, a bemeneti játékterület képéből kivágott golyók osztályozásáról lesz szó neurális hálózat segítségével. A kivágott képek fogják a bemenetet képezni, majd a neurális hálózat azt osztályozza egy 0 tól 7 ig terjedő egész számként. Ezek a számok a golyók színeit reprezentálják, lásd \ref{tab:golyo_azonositok} táblázat.

\begin{table}[!ht]
    \caption{A golyók szín szerint, és azok azonosítói.}
    \label{tab:golyo_azonositok}
	\footnotesize
	\centering
	\begin{tabular}{ l c }
		\toprule
		Golyó színe & Azonosító \\
		\midrule
		fekete      & 0\\
        kék         & 1\\
        barna       & 2\\
        zöld        & 3\\
        rózsaszín   & 4\\
        piros       & 5\\
        fehér       & 6\\
        sárga       & 7\\
		\bottomrule
	\end{tabular}
\end{table}

\par A neurális hálózat betanításához szükség van betanítási adatkészletre, viszont az előzőekben megsimert módszereknél szóba került, hogy viszonylag kevés kézi szortírozással könnyedén lehet velük előállítani adatkészletet, amely tökéletes a neurális hálózat betanításához és teszteléséhez. Az adatkészlet néhány eleme a \ref{fig:adatkeszlet} ábrán látható.

\begin{figure}[!ht]
    \centering
    \includegraphics[width=30mm, keepaspectratio]{figures/dataset_1.png}\hspace{2mm}
    \includegraphics[width=30mm, keepaspectratio]{figures/dataset_2.png}\hspace{2mm}
	\includegraphics[width=30mm, keepaspectratio]{figures/dataset_3.png}\\\vspace{2mm}
    \includegraphics[width=30mm, keepaspectratio]{figures/dataset_4.png}\hspace{2mm}
    \includegraphics[width=30mm, keepaspectratio]{figures/dataset_5.png}\hspace{2mm}
	\includegraphics[width=30mm, keepaspectratio]{figures/dataset_6.png}\\\vspace{2mm}
    \caption{Az adatkészlet elemei.}
    \label{fig:adatkeszlet}
\end{figure}

\par A betöltött adatok címkével megfelelően azonosítva átadásra kerülnek a neurális hálózatnak. Ahhoz, hogy a tanítás jó eredményeket hozzon, gondoskodni kell arról, hogy a betanítási adatok közt egyenlő arányban szerepelnek az egyes golyók színek szerint, továbbá, hogy az adatkészlet elemei megfelelően össze vannak keverve. Az előkészített adatkészlet egy kis része (10\% - 30\%) elkülönítésre kerül, amely a neurális hálózat pontosságának tesztelésére fog szolgálni.
\par A betanítási folyamatok után a neurális hálózat mentésre kerül, így az későbbi kívánt használatba vétel esetén egyszerűen betölthető, a tanítás és felhasználás külön programfájlokban könnyedén elvégezhető.
\par A körfelismerés módszerrel ötvözve, a neurális hálózattal való osztályozás gyors és pontos eredményeket biztosít. A felismerés egy kimenetele a \ref{fig:felismert_asztal} képen látható.

\begin{figure}[!ht]
    \centering
    \includegraphics[width=110mm, keepaspectratio]{figures/recognised_table.png}
    \caption{A neurális hálózattal való golyófelismerés kimenete.}
    \label{fig:felismert_asztal}
\end{figure}

\par A módszer eredményességének köszönhetően későbbiekben részletesebben ismertetem a megvalósítás fejezetben.

\section{A játékmenet elemzése}
\label{section:analyze_gameplay}
A golyók felismert pozíciója alapján a játékmenet többféle szempontból vizsgálható, ameyleknek köszönhetően érdekes statisztikai adatokhoz lehet jutni. A vizsgálati szempontok közül ebben a munkában négyfélét fogok megfigyelni, ezek az egyes golyók által \textbf{megtett távolság}, adott golyó \textbf{pillanatnyi sebessége}, a golyók \textbf{megtett útvonala} és a \textbf{zsebekben elhelyezett golyók megállapítása}.
\par Az említett szempontok alapján más komplexebb tényezőket is lehet vizsgálni (pl.: átlagos sebesség, golyók lepattanásának száma stb.), itt viszont az egyszerűség kedvéért csak az alapvetőbb szempontokat vizsgálom.
\par Ahhoz, hogy megtett utakat és sebességeket lehessen megállapítani, egy folytonos videófelvételre van szükség, a videófelvétel minősége és képkockasebessége is befolyásoló tényezők az adatok pontosságához és megállapításához.
\par A következő alfejezetekben az egyes vizsgálati szempontokat részletezem, viszont mindenek előtt még egy fontos problémának a megoldását írom le a következő fejezetben.

\subsection{A piros labdák felcímkézése}
\label{subsection:piros_labda_cimke}
A felismert golyók pozíciója és színe nagyszerű kiindulópont az egyes színű golyók útvonalának megállapításához, azonban a piros golyók sajátossága, hogy több is szerepel belőlük a játékterületen. Ez a tulajdonság ahhoz vezet, hogy az egyes golyókat meg kell különböztetni valahogy egymástól, hogy például azok útvonalát el lehessen tárolni.
\par A probléma megoldására létezik egy elméletben viszonylag egyszerű megoldás, amely a következő lépésekből áll:

\begin{itemize}
    \item 1. Első felismerés során az egyes piros golyók címkéjének tetszés szerinti eltárolása.
    \item 2. A következő felismerési folyamat során, az előzőleg és jelenleg felismert piros golyók egymástól viszonyított távolságának növekvő sorrendbe helyezése.
    \item 3. Az előző lépésben kiszámolt listán sorban végighaladva a jelenlegi golyókhoz a címkék beállítása, az előző golyók közül a hozzájuk legközelebb esővel.
\end{itemize}

\par A fenti módszer úgymond 'mohó' algoritmus, ezzel a számára legkedvezőbb feltételt választja minden esetben, feltételezve, hogy a golyók között a lehető legkisebb az összes elmozdulás. A módszer bizonytalannak tűnhet, azonban megfelelő minőségű videófelvétel mellett elfogadhatóan pontos eredményeket ad. A \ref{fig:felismert_asztal} ábrán látható felismert golyók ezen módszer segítségével lettek felcímkézve, és az alkalmazás megvalósításánál is a jelenlegi módszert alkalmazom.

\subsection{Megtett távolság és útvonal}
Ezt a két szempontot egy alfejezteben ismertetem, mert mindkettőjük megállapítása azonos és egyetlen alapon nyugszik, ez pedig nem más mint szimplán az adott golyók pozíciója.
\par A teljes megtett távolság könnyen megállapítható a golyók két képkocka közti elmozdulásainak a felhalmozásával. Ahhoz viszont, hogy egy hétköznapokban is gyakrabban használt mértékegység formájában mutatkozzanak a távolságok, egy apró trükkre lesz szükség. A golyók elmozdulásai kezdetben megállapításkor képpont formájában szerepelnek, a \ref{section:asztal_felismeres} alfejezet tábla felismerés részénél adott, hogy a felismeréshez használt asztal egy fix szélességre és magasságra van transzformálva. Ennek következtében megállapítható, hogy ha a transzformációs szélesség 1024 képpont és a szabvány snooker asztal mérete a \ref{section:snooker_altalanos} rész alapján 12 láb, avagy 365,8 cm, akkor a valóságban $1\text{ cm} = \frac{1024}{365,8} \approx 2.8\text{ képpont}$ formájában mérhető a transzformált képen.
\par A megtett útvonal megállapításánál nincs szükség mértékegyésg átváltásokra, hiszen az útvonal csak a felismerés közben a transzformált asztalon kerül kirajzolásra. Az útvonal a egyes képkockákon, adott golyók pozíciójának folytonos listában való eltárolásával lehetséges. A tárolt listák közül egyet a \ref{fig:golyo_utvonal} ábrán látható módon lehet megjeleníteni a transzformált képen.

\begin{figure}[!ht]
    \centering
    \includegraphics[width=110mm, keepaspectratio]{figures/ball_path.png}
    \caption{A fehér golyó útvonala a felismert képen kék vonallal jelölve.}
    \label{fig:golyo_utvonal}
\end{figure}

\subsection{A golyó sebessége}
Ahhoz, hogy a golyó sebességét meg lehessen állapítani, két tényezőre van szükség, ezek a megtett távolság és az ezen távolság megtételéhez szükséges idő. A megtett távolságot már az előző alfejezetben leírtak alapján megállapítható, tehát csak az eltelt időt kell kiszámolni a sebesség megállapításához. Az eltelt időt a felvétel egy paramétere, a képkockasebesség alapján fogom kiszámolni.
\par Ha a videófelvételről ismert, hogy a képkockasebessége 30 fps (másodpercenként 30 képkocka), vagy másnéven, két képkocka közt eltel idő $\frac{1}{30} \approx 0.03\text{ másodperc}$, és ismert, hogy két képkocka közt egy adott golyó elmozdulása például 3 képpont, vagy centiméterre átszámolva 1,07 cm, akkor a golyó sebessége $\frac{1,07}{0.03} \approx 35.67 \text{ cm/s}$.
\par A sebességek értékét egy listában tárolva, azokat a felismerési folyamat végén összegezve, majd leosztva a lista hosszával, kiszámolható a felismerés időtartama alatt egy adott golyóhoz tartozó átlagos sebesség.

\subsection{Zsebbe helyezés megállapítása}
A zsebbe helyezett golyó megállapítása ugyancsak a golyó pozíciójának a segítségéval állapítható meg, azonban közre játszik a feilsmert golyó beazonosításának megszakadása, megszűnése is. A golyó azonosítása akkor szűnik meg, amikor a golyó olyan takarásba vagy árnyékba kerül, hogy annak a felismerését nem lehet többé végrehajtani. Amikor egy golyó felismerése megszűnik, két képkocka közt a felismert golyók száma közt eltérés lép fel, ez visszavezethető egy adott golyóra, és annak az utolsó ismert pozíciójára.
\par Az eltűnt golyó utolsó pozíciójának ismeretében megvizsgálható annak a távolsága a zsebek pozíciójához viszonyítva. A zsebek pozíciója kiszámolható az előzőleg már említett transzformált kép tárolt fix szélesség és magasság értékeivel, hiszen a zsebek a négy sarkon és a hosszanti oldalak felénél helyezkednek el. A zsebek elhelyezkedését a \ref{tab:zsebek_kiszamolasa} táblázat szemlélteti.

\begin{table}[!ht]
    \caption{A zsebek elhelyezkedése szélesség és magasság alapján.}
    \label{tab:zsebek_kiszamolasa}
	\footnotesize
	\centering
	\begin{tabular}{ l l l l }
		\toprule
		(x, y)  & Bal           & Közép                                    & Jobb \\
		\midrule
		Alul    & (0, 0)        & ($\dfrac{\text{szélesség}}{2}$, 0)        & (szélesség, 0)\\
		Felül   & (0, magasság) & ($\dfrac{\text{szélesség}}{2}$, magasság) & (szélesség, magasság)\\
		\bottomrule
	\end{tabular}
\end{table}

\par Amennyiben a golyó utolsó ismert pozíciója elegendően közel van egy zseb pozíciójához, feltétlezhető, hogy a golyó az adott zsebben lett elhelyezve.
%\include{content/analizalas}
\chapter{A neurális hálózat betanítása}
Ebben a fejezetben részletezem a felismeréshez használt neurális hálózat betanítását, a betanításhoz a Tensorflow\cite{tensorflow_autodiff} könyvtárat fogom használni Python programozási nyelven. A betanítás után a neurális háló model egy fájlba kerül mentésre, amelyet a felismerő program betölt és használ golyók osztályozására.

\section{Könyvtárak importálása}
A betanításhoz szükség van néhány külső könyvtár használatára, ezek a \ref{cod:classifier_imports} kódrészletben láthatóak.

\begin{codewrapper}
\begin{lstlisting}[language=Python, numbers=left, caption={A betanításhoz használt könyvtárak importálása.}, label={cod:classifier_imports}]
import random
import numpy as np

import tensorflow as tf
from tensorflow import keras

import cv2
import os
\end{lstlisting}
\end{codewrapper}

\par A \lstinline{random} és \lstinline{numpy} könyvtárak az adatkészlet betöltéséhez kerülnek felhasználásra főként, a \lstinline{numpy} könyvtárral lehetséges nagyméretű tömbökön műveletek gyors végrehajtása, mátrix műveletek elvégzése, a \lstinline{random} könyvtárral a betöltött adatokat lehet keverni, ami fontos a háló megfelelő betanításához.
\par Szükség van még az előzőekben említett Tensorflow könyvtár importálására, ez az 5. és 6. sorban történik meg, továbbá az OpenCV könyvtárra a képek betöltéséhez és HSV konvertálásához. Végül, a mappák eléréséhez az \lstinline{os} könyvtár kerül felhasználásra.

\section{Az adatkészlet betöltése}
Az adatkészlet betöltésénél a cél, hogy legyen egy betanítási és egy tesztelési adatkészlet, amelyek felépülnek egy golyóhoz tartozó kép adataiból és egy hozzájuk tartozó címkéből, amely a golyó színét adja meg. A tanítási adatkészlet a nevéből adódóan a hálózat betanításához használatos, a tesztelési adatkészlet a betanítási folyamat ellenőrzésére szolgál.
\par Ahhoz hogy el lehessen készíteni ezeket az adatkészleteket először be kell tölteni az adatokat. Az adatok színenként csoportosított mappákban helyezkednek el, azomban egyes mappákban nem egyenlő mennyiségű adat áll rendelkezésre, ezért az adatok betöltése előtt meg kell állapítani a lekevesebb adatot tartalmazó mappát, majd ez alapján betölteni a többit. A méret megállapítását a \ref{cod:smallest_dataset} kódrészlet mutatja.

\begin{codewrapper}
\begin{lstlisting}[language=Python, numbers=left, caption={A legkevesebb elemmel rendelkező adatkészlet megállapítása.}, label={cod:smallest_dataset}]
directory = "./dataset"
classes = ["black", "blue", "brown", "green", "pink", "red", "white", "yellow"]

lengths = []
for class in classes:
    class_dir = directory + "/" + class
    files = os.listdir(class_dir)
    lengths.append(len(files))

size = min(lengths)
\end{lstlisting}
\end{codewrapper}

\par Ebben a kódrészletben először meg kell adni az elérési útvonalat a \lstinline{directory} változóval, továbbá az egyes színekhez tartozó mappák listáját a \lstinline{classes} listával. A \lstinline{lengths} lista fogja tárolni az egyes mappákban található adatok mennyiségét.
\par A mennyiségek kiszámolásához először a \lstinline{class_dir} változóba belekerül a tényleges elérési útvonal, ez például a \lstinline{classes} tömb 0. elemének esetében '\lstinline{./dataset/black}' értéket vesz fel. A mappában található fájlok listáját a \lstinline{os.listdir} függvény adja meg az elérési útvonal alapján, majd a listát beleteszi a \lstinline{files} tömbbe. A fájlok mennyisége ezután a \lstinline{lengths} listához adódik hozzá a kód 8. sorában látható módon. A mennyiség lista legkisebb elemét a \lstinline{min} függvény adja meg, az eredmény a \lstinline{size} változóban kerül tárolásra.
\par A legkisebb adatkészlet megállapítása után következik az adatok betöltése, a betöltéshez használt kód a \ref{cod:dataset_load} kódrészletben látható.

\begin{codewrapper}
\begin{lstlisting}[language=Python, numbers=left, caption={Az adatkészlet betöltése.}, label={cod:dataset_load}]
images = []
labels = []

for class in classes:
    class_dir = directory + "/" + class
    files = os.listdir(class_dir)

    random.shuffle(files)
    files = files[:size]

    for file in files:
        bgr = cv2.imread(class_dir + "/" + file, cv2.IMREAD_COLOR)
        hsv = cv2.cvtColor(bgr, cv2.COLOR_BGR2HSV)

        combined = np.append(bgr, hsv, axis=2)
        images.append(combined)
        labels.append(np.where(classes == class))
\end{lstlisting}
\end{codewrapper}

\par Itt a fájlok listájának megszerzéséig a folyamat megegyezik a \ref{cod:smallest_dataset} kódrészletben használt módszerrel, a különbségek a 8. sortól kezdődnek, itt a \lstinline{random.shuffle} függvény segítségével a \lstinline{files} tömb összekeverése történik meg, majd a tömb mérete levágásra kerül az előzőekben kiszámolt legkisebb adatkészlet méretére. A levágás után a szkript végighalad a fájlokon, azokat betöltve a \lstinline{cv2.imread} függvény\cite{opencv_docs} segítségével, amelynek első paramétere a betöltendő kép elérési útja, a második pedig a betöltés módja, amely jelen esetben \lstinline{cv2.IMREAD_COLOR}, ez azt adja meg, hogy a kép színes BGR formátumban kerüljön betöltésre.
\par A betöltött kép a \lstinline{bgr} változóba kerül, a képet a felismeréshez konvertálni kell HSV formátumra, ezt a \lstinline{cv2.cvtColor} függvény teszi meg, amelynek meg kell adni az átalakítani kívánt képet és az átalakítás módjat (\lstinline{cv2.COLOR_BGR2HSV}), amely jelen esetben a HSV konverziót jelzi. A HSV konverzióról bővebben a \ref{section:megv_asztal_kontur} részben lesz szó.
\par A konverzió után a BGR és HSV értékek intenzitás értékei összefűzésre kerülnek a \lstinline{np.append} függvény segítségével, itt a képek megadásával meg kell adni, hogy az összefűzés a képek mátrixának 3. dimenzióján történjen, ezt a \lstinline{axis=2} paraméter adja meg. Az összefűzés után a kezdetben létrehozott \lstinline{images} tömbbe kerül bele az összefűzőtt kép, a hozzá tartozó címke pedig a \ref{cod:smallest_dataset} kódrészletben található \lstinline{classes} tömbben található index alapján kerül hozzáfűzésre a \lstinline{labels} tömbhöz.
\par A következő lépés a betanítási és tesztelési adatkészletek különválasztása, ennek a folyamata a \ref{cod:train_test_separate} kódrészletben található.

\begin{codewrapper}
\begin{lstlisting}[language=Python, numbers=left, caption={A tanítási és tesztelési adatkészletek elkülönítése.}, label={cod:train_test_separate}]
images = np.array(images)
labels = np.array(labels).reshape(-1)

labels = keras.utils.to_categorical(labels)

border = int(size * 0.8)

train_images = images[:border]
train_labels = labels[:border]
test_images = images[border:]
test_labels = labels[border:]
\end{lstlisting}
\end{codewrapper}

\par A kódrészletben először az \lstinline{images} és \lstinline{labels} változókat numpy tömbre alakítjom a \lstinline{np.array} függvény segítségével, majd a \lstinline{labels} tömböt átalakítom bináris osztály mátix formájába a \lstinline{keras.utils.to_categorical} függvény\cite{tensorflow_docs} segítségével. Ez a konvertálás mindössze annyit jelent, hogy egy adott címke például \lstinline{3} bináris reprezentációban lesz feltüntetve egy \lstinline{[0, 0, 0, 1, 0, 0, 0, 0]} tömb formájában, feltéve ha 8 db címke áll rendelkezésre. Ezt az átalakítást szokták one-hot kódolásnak is nevezni\cite{harris2012digital}. Az egyes címkék kódolt változatai a \ref{tab:one_hot_labels} táblázatban láthatóak.

\begin{table}[!ht]
    \caption{A golyók szín szerint, és azok azonosítói.}
    \label{tab:one_hot_labels}
	\footnotesize
	\centering
	\begin{tabular}{ l c }
		\toprule
		Golyó címkéje & One-hot kódolt reprezentáció \\
		\midrule
		fekete      & [1, 0, 0, 0, 0, 0, 0, 0]\\
        kék         & [0, 1, 0, 0, 0, 0, 0, 0]\\
        barna       & [0, 0, 1, 0, 0, 0, 0, 0]\\
        zöld        & [0, 0, 0, 1, 0, 0, 0, 0]\\
        rózsaszín   & [0, 0, 0, 0, 1, 0, 0, 0]\\
        piros       & [0, 0, 0, 0, 0, 1, 0, 0]\\
        fehér       & [0, 0, 0, 0, 0, 0, 1, 0]\\
        sárga       & [0, 0, 0, 0, 0, 0, 0, 1]\\
		\bottomrule
	\end{tabular}
\end{table}

\par A kódolást követően egy határérték kerül kiszámolásra a \lstinline{border} változóba, ez szimplán megadja, hogy az elemek mekkora része legyen tanításra és mekkora legyen tesztelésre felhasználva. A határérték egy egész szám formájában adja meg az elválasztás pontját, majd a szétválasztás a 8. - 11. sorban látható módon megy végbe. Az elválasztás végeztével az adatkészlet készen áll a betanítási folyamatokra.

\section{A neurális hálózat felépítése és betanítása}
A neurális hálózat több rétegből épül fel, ezek közül az első réteg egy konvolúciós réteg. Ez a konvolúciós réteg közvetlenül fogadja a felismerésre szánt képet, amely jelen esetben egy 5 x 5 méretű kernelt használ 8 különböző konvolúció elkészítéséhez. A konvolúciók elkészítése után egy ún. Max-Pool eljárással azok léképezése történik, az eljárás egy 2 x 2 méretű kernellel dolgozik, minden ilyen kernel elemeit kombinálja egy értékbe, ennek következtében a konvolúciók mátrixának mérete felére csökken. Ez a művelet a teljesítmény növelésében játszik szerepet a neurális hálózat kapcsolatainak leszűkítésével.
\par A konvolúció és Max-Pool rétegek után az adatok egy ún. 'lapítás' (flatten) műveleten esnek át ez az elemeket egy egydimenziós vektorba helyezi. A vektorrá alakítás után következik még két, egy 18 és egy 8 csomópontból álló réteg, ezek a lapított réteghez hasonlóan szintén egydimenziós vektorok. A két réteg méretét próbálgatás útján állapítottam meg a megfelelő sebesség és pontosság eléréséhez.
\par Az utolsó réteg egy kimeneti réteg, amely megadja a bemenetként adott golyóról, hogy azt egyes színek milyen mértékben reprezentálják. Ez egy 8 elemű \ref{tab:one_hot_labels} táblázatban látott one-hot értékhez hasonló lista formájában mutatkozik. A neurális hálózat felépítését a \ref{fig:neural_network_model} ábra szemlélteti, hálózat konfigurálásához használt kód pedig a \ref{cod:neural_network_model_def} kódrészletben látható.

\begin{figure}[!ht]
    \centering
    \includegraphics[width=140mm, keepaspectratio]{figures/neural_network_model.png}
    \caption[A neurális hálózat modeljének felépítése.]{A neurális hálózat modeljének felépítése.\cite{alexlenail}}
    \label{fig:neural_network_model}
\end{figure}

\begin{codewrapper}
\begin{lstlisting}[language=Python, numbers=left, caption={A neurális hálózat konfigurálása.}, label={cod:neural_network_model_def}]
model = keras.models.Sequential([
    keras.layers.Conv2D(8, (5, 5), activation="relu", input_shape=(18, 18, 6)),
    keras.layers.MaxPool2D((2, 2)),
    keras.layers.Flatten(),
    keras.layers.Dense(18, activation="relu"),
    keras.layers.Dense(8, activation="relu"),
    keras.layers.Dense(8)
])
\end{lstlisting}
\end{codewrapper}

\par A \ref{cod:neural_network_model_def} kódrészletben az első sorban a model létrehozása történik, ezt a \lstinline{keras.models.Sequential} függvény\cite{chollet2015keras,tensorflow_docs} teszi meg. A modell a \lstinline{model} változóban kerül tárolásra. A létrehozáshoz használt függvény belsejében található lista az egyes rétegeket adja meg, ezek közül az első a konvolúciós réteg, amelyet a \lstinline{keras.layers.Conv2D} függvény\cite{chollet2015keras,tensorflow_docs} ad meg. A függvény paraméterei megadják a konvolúciók számát, a használt kernel méretét, az aktivációs függvényt és a bemeneti mátrix alakját. Az aktivációs függvény megadja, hogy adott bemenetekre hogyan reagáljon, és milyen kimenetet adjon a neurális hálózat egy csomópontja\cite{hinkelmannneural}. Ennél a megoldásnál egy ún. ReLU (Rectified Linear Unit) aktivációs függvény kerül felhasználásra, amelyet a \ref{for:RELU} egyenlet\cite{RELU2010} ír le.

\begin{equation}
    f(x) =
    \begin{cases}
        0 & ,\text{ha}\ x\le0 \\
        x & ,\text{ha}\ x>0
    \end{cases}
    \label{for:RELU}
\end{equation}

\par Az egyenlet alapján, ha az aktivációs függvény bemeneti értéke 0 alatt van, akkor a kimeneti 0, ha 0 felett van, akkor pedig a bemeneti érték megegyezik a kimeneti értékkel. A függvényt a \ref{fig:relu_function} ábra mutatja koordináta rendszerben.

\begin{figure}[!ht]
    \centering
    \includegraphics[width=100mm, keepaspectratio]{figures/relu_function.png}
    \caption{ReLU aktivációs függvény ábrázolása.}
    \label{fig:relu_function}
\end{figure}

\par A \ref{cod:neural_network_model_def} kódrészlethez visszatérve a hálózat második rétege a \lstinline{keras.layers.MaxPool2D} függvénnyel\cite{chollet2015keras,tensorflow_docs} kerül megadásra, itt az egyetlen paraméter a kernel mérete. A max-pool réteg után következik a lapítást végző réteg, ezt a \lstinline{keras.layers.Flatten} függvény\cite{chollet2015keras,tensorflow_docs} adja meg, majd ezután következik még két teljesen kapcsol (Dense) réteg, amelyeket a \lstinline{keras.layers.Dense} függvény\cite{chollet2015keras,tensorflow_docs} biztosít. A teljesen kapcsolt rétegek függvényének bemenetei a réteg nagysága és aktivációs függvénye, ami ebben az esetben is a ReLU aktivációs függvény. Az utolsó kimeneti réteg is az előzőekhez hasonló teljesen kapcsolt réteg, viszont itt nincs szükség aktivációs függvényre.
\par A neurális hálózat rétegeinek megadása után be kell állítani a veszteségfüggvényt és az optimalizáló algoritmust, majd ezután megezdődhet a háló betanítása és kiértékelése az adatkészletek segítségével. Az említett konfigurálás programkód formájában a \ref{cod:train_network} kódrészletben szerepel.

\begin{codewrapper}
\begin{lstlisting}[language=Python, numbers=left, caption={A neurális hálózat betanítása.}, label={cod:train_network}]
loss = keras.losses.CategoricalCrossentropy(from_logits=True)
optimizer = keras.optimizers.Adam(learning_rate=0.0001)

model.compile(loss=loss, optimizer=optimizer)
model.fit(train_images, train_labels, epochs=20, batch_size=5)
model.evaluate(test_images, test_labels, batch_size=5)

model.add(keras.layers.Softmax())

model.save("classifier.h5")
\end{lstlisting}
\end{codewrapper}

\par Itt a veszteségfüggvényt a \lstinline{keras.losses.CategoricalCrossentropy} függvény adja meg\cite{chollet2015keras,tensorflow_docs}, ami kereszt entrópiát\cite{rubinstein2004cross} használ a veszteség kiszámolásához. A veszteség ezesetben megadja, hogy egy adott bemenet alapján a kimeneti érték mennyiben különbözik az elvárt kimeneti értéktől. Az optimalizáló algoritmus a \lstinline{keras.optimizers.Adam} függvénnyel\cite{chollet2015keras,tensorflow_docs} adható meg, ez egy gyakran használt optimalizáló algoritmust alkalmaz, amelyet Adam optimalizálónak neveznek, ez az algorimus egy gradiens süllyedéses módszerrel közelíti a kimeneti értéket az elvárt értékhez a veszteségfüggvény segítségével\cite{adam2014}. A \lstinline{learning_rate} paraméterrel megadható a tanítás sebessége, amely a gradiens süllyedés módszer lépéseinek nagyságával befolyásolja a sebességet. Túl nagy tanítási sebesség a pontosság veszítéséhez vezethet, azomban túl kicsi sebesség viszont a tanításhoz szükséges időt nagymértékben növelheti.
\par A veszteség függvény és optimalizáló beállítása után a hálózat fordítását a \lstinline{model.compile} függvénnyel\cite{chollet2015keras,tensorflow_docs} lehet elvégezni, amelynek paraméterként meg kell adni a veszteségfüggvényt és optimalizálót, ezután a betanítást a \lstinline{model.fit} függvénnyel\cite{chollet2015keras,tensorflow_docs} lehet megtenni, amelynek meg kell adni a betanításhoz szükséges adatokat és azok címkéjeit, továbbá meg kell adni egy \lstinline{epochs} paramétert, ez megadja, hogy tanítás során az algoritmus hányszor iteráljon végig a teljes adatkészleten, továbbá meg kell adni egy \lstinline{batch_size} paramétert, ami megadja, hogy mekkora adatcsoportonként frissítse az értékeit a gradiens süllyedéses algoritmus. Tanítás közben a hálózat vesztesége, avagy a kapott eredmény és várt eredmény eltérése folyamatosan csökken, ezt a \ref{fig:loss_metrics} ábra szemlélteti.

\begin{figure}[!ht]
    \centering
    \includegraphics[width=110mm, keepaspectratio]{figures/loss_metrics.png}
    \caption{A veszteség alakulása a tanítási folyamat során.}
    \label{fig:loss_metrics}
\end{figure}

\par A tanítás után a kiértékelést a \lstinline{model.evaluate} függvénnyel\cite{chollet2015keras,tensorflow_docs} lehet elvégezni, ennek bemenetként a teszt adatkészlet és címkék kerülnek megadásra, továbbá a tanításhoz hasonlóan az adatcsoportok mérete.
\par A betanítás után a modelhez egy Softmax réteg kerül hozzáadásra a programkód 8. sorában látható módon, ez a réteg a kimeneti értékeket átalakítja valószínűség formájába, ami azt jelenti, hogy egy jósolt érték 8 elemből álló listájánál az értékek összege mindíg 1 értéket ad.
\par A hálózat betanítása után annak elmentése egy fájlba a \lstinline{model.save} függvénnyel\cite{chollet2015keras,tensorflow_docs} lehetséges, amelynek paraméterül a mentéshez használt nevet kell megadni. A mentés után a modellt a felismerő programba könnyedén be lehet tölteni, és osztályozni a golyókat színük alapján.
\renewcommand{\lstlistingname}{kódrészlet}

\chapter{A felismerő megvalósítása}
\section{Alkalmazott módszerek}
A felismerő program megvalósítása során az eddig megismert módszereket fogom felhasználni, azokat C++ programozási nyelven fogom elkészíteni és ismertetni. Az egyes algoritmusokat függvények formájában készítem el, ezeket a függvényeket pedig több helyen is felhasználom.
\par Az eddig megismert módszerek közül elsősorban a nyers bemenetből emelem ki a játékterületet, ezt követően a golyók pozíciójának felismeréséhez kör detektálást és egy neurális hálózatot fogok használni. A következőkben az egyes függvények működését, azokban felhasznált külső könyvtárak eszközeit ismertetem részleteiben.
\par A fejezetek felosztása az eddig megismert lépések szerint kerül rendezésre.

\section{A szükséges könyvtárak importálása}
Ahhoz hogy a függvények megfeleően működjenek, meg kell mondani a programnak, hogy használja a külső könyvtárakat.
\newline A legfontosabb könyvtárak importálása a \ref{cod:import} kódsorok alapján tehető meg.

\vspace{2mm}
\hspace{-10mm}
\begin{minipage}{\linewidth}
\begin{lstlisting}[language=C++, numbers=left, caption={Könyvtárak importálása.}, label={cod:import}]
#include <opencv2/opencv.hpp>
#include <fdeep/fdeep.hpp>
\end{lstlisting}
\end{minipage}

\par Az \lstinline{opencv.hpp} könyvtár az OpenCV által biztosított képfeldolgozási függvényeket biztosítja, a \lstinline{fdeep.hpp} könyvtár pedig a Tensorflow -al készített neurális hálózat betöltését teszi lehetővé.

\section{Az asztal kontúrjának megkeresése}
\label{section:megv_asztal_kontur}
A nyers képből a játékterület megszerzéséhez azt először be kell tölteni egy többdimenziós tömbbe. A kép betöltése többféleképp végbemehet, ezért ezt konkrétan nem részletezem.
\par A betöltött kép tömbjének alakja megegyezik a kép szélességével és magasságával, továbbá az intenzitási értékekkel, tehát egy 1024 x 512 méretű RGB képet betöltve, annak tömbjének az első és második dimenziója 1024 és 512, a harmadik pedig az RGB (Piros, Zöld, Kék) intenzitásoknak megfelelően 3 méretű.
\par Fontos megjegyezni, hogy az OpenCV a képeket betöltéskor BGR formátumban tölti be, ez az elnevezésből adódóan annyiban tér el az RGB formátumtól, hogy a piros (R) és kék (B) színcsatornák fel vannak cserélve.
\par A nyers bemeneti kép megszerzése után következik az asztal kontúrjának megkeresése. Első lépésként a képet HSV formátumra kell alakítani, majd az alsó és felső intenzitási értékhatárok megadásával meghatározható a maszk, amely alkalmazható az eredeti képre.
\newline A fentieket a \ref{cod:maszk} kódrészlettel végzem el.


\vspace{2mm}
\hspace{-10mm}
\begin{minipage}{\linewidth}
\begin{lstlisting}[language=C++, numbers=left, caption={A játékterület maszkolása.}, label={cod:maszk}]
cv::cvtColor(image, hsv, cv::COLOR_BGR2HSV);

cv::Scalar lowerGreen = {50, 50, 70};
cv::Scalar upperGreen = {65, 255, 255j};

cv::inRange(hsv, lowerGreen, upperGreen, mask);
cv::bitwise_and(image, image, result, mask);
\end{lstlisting}
\end{minipage}

\par A \ref{cod:maszk} kódrészletben az \lstinline{image} a bemeneti kép, amelyet a \lstinline{cv::cvtColor} függvénnyel \cite{opencv_docs} konvertálok át HSV formátumra. Ennek a függvénynek az első paramétere a bemeneti kép, a második a kimeneti kép változója, a harmadik pedig a konverzió típusa, amely ebben az esetben BGR $\rightarrow$ HSV.
\par A BGR értékekből a HSV értékek kiszámolásához először az értéket (Value) kell kiszámolni, ez a \ref{for:HSV_V} egyenlet\cite{opencv_docs} szerint lehetséges,

\begin{equation}
    V \leftarrow max(R,G,B)
    \label{for:HSV_V}
\end{equation}

\par ahol $V$ az értéket (Value) jelöli, $R$, $G$ és $B$ pedig az adott képpont három színkomponensét (Piros, Zöld, Kék). Az egyenlet alapján az érték a három színkomponens közül a legnagyobbnak az értékét fogja felvenni. Az érték kiszámolásával megadható a telítettség (Saturation).
\par Ezt a \ref{for:HSV_S} egyenlet\cite{opencv_docs} alapján lehet kiszámolni,

\begin{equation}
    S \leftarrow
    \begin{cases}
        \frac{V-min(R,G,B)}{V} & ,\text{ha}\ V\neq0 \\
        0 & ,\text{különben}
    \end{cases}
    \label{for:HSV_S}
\end{equation}

\par itt $S$ a telítettséget (Saturation) jelöli, és az előzőekhez hasonlóan $V$ az értéket (Value), $R$, $G$ és $B$ pedig a színkomponenseket, továbbá $min(R,G,B)$ a három színkomponens közül a legkisebbet adja meg.
\par Az árnyalatot (Hue) szintén az érték (Value) segítségével lehet kiszámolni, ezt a \ref{for:HSV_H} egyenlet\cite{opencv_docs} adja meg,

\begin{equation}
    H \leftarrow
    \begin{cases}
        \frac{60(G-B)}{V-min(R,G,B)} & ,\text{ha}\ V=R \\[5pt]
        \frac{120+60(B-R)}{V-min(R,G,B)} & ,\text{ha}\ V=G \\[5pt]
        \frac{240+60(R-G)}{V-min(R,G,B)} & ,\text{ha}\ V=B \\[5pt]
        0 & ,\text{ha}\ R=G=B
    \end{cases}
    \label{for:HSV_H}
\end{equation}

\par ahol $H$ az árnyalatot (Hue), $V$ az értéket (Value), $R$, $G$ és $B$ a három színkomponenst, $min(R,G,B)$ pedig a színkomponensek közül a minimálisat jelöli.
\par Amennyiben $H$ értéke kisebb, mint $0$, annak értéke $H \leftarrow H+360$ szerint alakul. A 8-bites és 16-bites színnel rendelkező képeknél $R$, $G$ és $B$ értéke kezdetben normalizálásra kerül a $[0,1]$ intervallumba, ennek következtében a három értéknél $0 \le V \le 1$, $0 \le S \le 1$, $0 \le H \le 360$ tartományok jelentkeznek. Az értékek visszaállítása tartománynak megfelelően 8-bites képek esetében az értékek megszerzése után $V \leftarrow 255V$, $S \leftarrow 255S$ és $H \leftarrow H/2$ szerint megy végbe, ez hasonló 16-bites szín esetében is. 32-bites színnel rendelkező képeknél nincs kezdeti normalizálás, és ennek következtében visszaalakítás sem szükséges.\cite{opencv_docs}

\par Az \ref{cod:maszk} kódrészlethez visszatérve, a \lstinline{lowerGreen} és \lstinline{upperGreen} változók az alsó és felső intenzitási határokat jelölik sorrendnek megfelelően. A maszk elkészítését a \lstinline{cv::inRange} függvénnyel \cite{opencv_docs} végzem el, itt a paraméterek sorban a HSV re konvertált kép, az alsó és felső intenzitás értékek, valamint a kimeneti maszk változója.
\newline A függvény a \ref{for:maszkolas} egyenlet alapján dönti el, a maszk intenzitását,

\begin{equation}
    M(I) = L(I) \le S(I) \le U(I)
    \label{for:maszkolas}
\end{equation}

\par ahol $M$ a maszk, $L$ az alsó, $U$ a felső és $S$ a bemeneti HSV képet jelöli. A \ref{for:maszkolas} függvény mindhárom intenzitásra alkalmazásra kerül, a maszkban az intervallumon belüli intenzitások 255, a kívüliek pedig 0 értéket kapnak. A maszk elkészítése után a maszkolás megtörténik az eredeti bemenő képre a \lstinline{cv::bitwise_and} függvény \cite{opencv_docs} segítségével. Itt a paraméterek a bejövő eredeti kép \lstinline{image} kétszer, a kimeneti kép és a maszk \lstinline{mask}.
\newline A folyamat során a metódus a \ref{for:maszk_alkalmazas} egyenlet szerint jár el,

\begin{equation}
    R(I) = S_1(I)\quad \land\quad S_2(I)\qquad ,ha\quad M(I) \ne 0
    \label{for:maszk_alkalmazas}
\end{equation}

\par ahol $R$ a kimenő maszkolt kép (\lstinline{result}) $S_1$ és $S_2$ a két bemeneti kép paraméter, és $M$ a maszk. A bemenetben a kép azért szerepel kétszer egymás után, mert a \ref{for:maszk_alkalmazas} függvényben láthatóan a két bemenő paraméter közt egy bit szintű 'és' művelet történik, amennyiben a maszk nem nulla. Ennek eredményeképp az eredeti kép adódik vissza amelyen maszkolt képpontok feketével szerepelnek. Ez azért történik, mert bit szinten ha két megegyező elem közt történik 'és' művelet, akkor az eredmény szintén megegyezik a két elemmel. Ennek a folyamatnak a kimenetele látható a már előzőleg tárgyalt \ref{fig:bemeneti_kep_mask} ábrán.
\par A maszkolt kép megszerzése után elvégezhető az éldetektálás, amelyet megelőz egy szürkeárnyalatolás.

\vspace{2mm}
\hspace{-10mm}
\begin{minipage}{\linewidth}
\begin{lstlisting}[language=C++, numbers=left, caption={Szürkeárnyalatolás és éldetektálás.}, label={cod:gray_and_canny}]
cv::cvtColor(result, imageGray, cv::COLOR_BGR2GRAY);

cv::Canny(imageGray, edges, 200, 100);
\end{lstlisting}
\end{minipage}

\par A szürkeárnyalati konverziót a már megismert \lstinline{cv::cvtColor} függvénnyel \cite{opencv_docs} végzem el a \ref{cod:gray_and_canny} kódrészlet alapján, majd ezután megkeressem az éleket a képen Canny éldetektálás \cite{opencv_docs, canny_edge_detection} (\lstinline{cv::Canny}) segítségével.
\newline A Canny éldetektálás általában több lépésre bontható szét, ezek lehetnek:

\begin{itemize}
    \setlength\itemsep{-2pt}
    \item homályosítás Gauss szűrővel \cite{shapiro2001} a zajcsökkentés érdekében,
    \item élek helyének és irányának megállapítása intenzitás-gradiensből,
    \item nem-maximum vágás merőleges élek szűréshéhez,
    \item kettős küszöbölés élek szűréséhez.
\end{itemize}

\par Az éldetektálásnál meg kell adni a függvénynek a szürkeárnyalatos képet (\lstinline{imageGray}), a kimeneti képet, továbbá két küszöbértéket, amelyet a Canny detektálás a kettős küszöbölés folyamat során fog felhasználni. Itt, ha a felső küszöb felett van egy potenciális él, az hozzáadódik az élek közé, ha az alsó küszöb alatt van eldobódik és ha a felső és alsó küszöbök közt helyezkedik el, akkor a szomszédos pixelek alapján kerül az élek közé. Az éldetektálással kapott kép (\lstinline{edges}) a \ref{fig:bemeneti_kep_edge} ábrán látható.

\begin{figure}[!ht]
    \centering
    \includegraphics[width=140mm, keepaspectratio]{figures/input_screen_edge.png}
    \caption{A Canny éldetektálás után kapott kép.}
    \label{fig:bemeneti_kep_edge}
\end{figure}

\par A következő lépésben a bináris képen lefuttatásra kerül egy kontúrkereső algoritmus \cite{SUZUKI198532}, majd a kapott kontúroknak vesszem a konvex körvonalát, azok egyszerűsítése, esetleges konkáv alakzatok megszüntetése érdekében. Ezek után feltételezve, hogy a kontúrok közül a legnagyobb a játékterület, az kiválasztható a körvonalak közül.

\vspace{2mm}
\hspace{-10mm}
\begin{minipage}{\linewidth}
\begin{lstlisting}[language=C++, numbers=left, caption={Kontúrok keresése.}, label={cod:contours}]
cv::findContours(edges, contours, cv::RETR_LIST, cv::CHAIN_APPROX_SIMPLE);

for(auto& contour : contours) {
    std::vector<cv::Point> hull;
    cv::convexHull(contour, hull);
    contour = hull;
}

std::sort(contours.begin(), contours.end(), areaComparator);
\end{lstlisting}
\end{minipage}

\vspace{2mm}
\hspace{-10mm}
\begin{minipage}{\linewidth}
\begin{lstlisting}[language=C++, numbers=left, caption={Sorba rendezéshez használt segédfüggvény.}, label={cod:contours_helper}]
bool areaComparator(const std::vector<cv::Point>& lhs, const std::vector<cv::Point>& rhs) {
    return cv::contourArea(lhs) > cv::contourArea(rhs);
}
\end{lstlisting}
\end{minipage}

\par A \ref{cod:contours} kódrészletben található \lstinline{cv::findContours} függvény \cite{opencv_docs, SUZUKI198532} egy határkövetéses algoritmussal kigyűjti a kontúrokat. Ezek a kontúrok a képen található képpont koordináták láncolatából állnak össze. A kontúrok körvonala a \lstinline{cv::convexHull} függvény \cite{opencv_docs, SKLANSKY198279} segítségével kapható meg. Ez az algoritmus a kontúrok koordinátáinak láncolatát használja, majd a kontúrt egy konvex körvonallal határolja, ugyancsak koordináták láncolatai formájában reprezentálva.
\par A fenti művelet elsőre feleslegesnek tűnhet, hiszen a keresett asztal kontúrja előreláthatólag nem konkáv, a művelet elvégzése mégis fontos, hiszen így egyszerűsíthető az alakzat (kontúr koordináta láncolat pontjainak csökkentése), ezzel a folytatólagos műveleteket felgyorsítva.
\par A legnagyobb kontúr kiválasztásához tudni kell az egyes kontúrok területét. A területet a \lstinline{cv::contourArea} függvénnyel \cite{opencv_docs} lehet kiszámolni. Ez megtehető minden eddigi kontúr esetében függvénynek való paraméterkénti átadással. A kiszámolt területek közül a legnagyobbat kiválasztva, annak kontúr koordináta láncolata eltárolásra kerül. A sorba rendezés a \ref{cod:contours} kódrészlet utolsó sorában látható függvénnyel történik, itt az \lstinline{areaComparator} egy segédfüggvény, amely az előzőleg említett \lstinline{cv::contourArea} metódust\cite{opencv_docs} használja a területek összehasonlításához és rendezéséhez. A segédfüggvény a \ref{cod:contours_helper} kódrészletben látható. A kapott kontúr kirajzolva a \ref{fig:bemeneti_kep_contour} ábrán látható.

\begin{figure}[!ht]
    \centering
    \includegraphics[width=140mm, keepaspectratio]{figures/input_screen_contour.png}
    \caption{A felismert asztal kontúrja a bináris képen, piros körvonallal keretezve.}
    \label{fig:bemeneti_kep_contour}
\end{figure}

\par A \lstinline{cv::contourArea} függvény a Surveyor's Area algoritmust \cite{braden1986surveyor} használja az alakzatok területének számolásához. Ez az algoritmus a Green-tétel egy speciális esete, amely alkalmazható egyszerű sokszögekre.
\newline Az algoritmus a \ref{for:green_formula} egyenletben látható,

\begin{equation}
    A = \sum^n_{k=0}\frac{(x_{k+1} + x_k)(y_{k+1} - y_k)}{2}
    \label{for:green_formula}
\end{equation}

\par ahol $n$ az óramutató járásával ellentétesen rendezett kontúr koordináták száma, $(x_k, y_k)$ a $k$ adik koordináta $x$ és $y$ pozíciója, és feltételezhető, hogy a $k = n+1$ elem megegyezik a $k = 0$ elemmel.

\par A \ref{fig:bemeneti_kep_contour} ábrán látható, hogy a kontúr téglalaphoz hasonló alajkának ellenére több, mint 4 pontból áll. Ahhoz hogy téglalap formájában legyen kivágva a kép, meg kell keresni azt a négyszöget, amely a kontúrt határolja. Erre egy olyan algoritmust készítettem, amely megkeresi a kontúr koordináták segítségével a négy leghosszabb oldalt, majd kiszámolja ezek metszéspontját. A négy leghosszabb oldal használata feltételezi, hogy a kép közel felső nézetből készült az asztalró, továbbá, hogy a sarkoknál jelenik meg több pont a kontúr keresés után.
\par Az oldalhosszok számolása a \ref{for:vector_distance} képlet alapján megy végbe,

\begin{equation}
    D = \sqrt{(x_a-x_b)^2 + (y_a-y_b)^2}
    \label{for:vector_distance}
\end{equation}

\par ahol $D$ a kiszámolt pontok közti távolság, $(x_a,y_a)$ és $(x_b,y_b)$ pedig a két koordináta, ameyek közt a táv számolandó. Miután megvannak az oldalak hosszai, eltárolásra kerül a négy legnagyobb oldalhoz tartozó koordináta. A kiválasztott pontoknál fontos, hogy óramutató járásával ellentétes sorrendben legyenek rendezve, amennyiben nem, a négyszög kontúr később hibás lehet.
\par A metszéspontok kiszámolásához a következő képleteket \cite{line_line} használtam,

\begin{equation}
    D = (x_1 - x_2)(y_3 - y_4) - (y_1 - y_2)(x_3 - x_4)
    \label{for:vector_intersection_denominator}
\end{equation}
\begin{equation}
    P_x = \frac{(x_1y_2 - y_1x_2)(x_3 - x_4) - (x_3y_4 - y_3x_4)(x_1 - x_2)}{D}
    \label{for:vector_intersection_point_x}
\end{equation}
\begin{equation}
    P_y = \frac{(x_1y_2 - y_1x_2)(y_3 - y_4) - (x_3y_4 - y_3x_4)(y_1 - y_2)}{D}
    \label{for:vector_intersection_point_y}
\end{equation}

\par ahol a \ref{for:vector_intersection_denominator} képletben a $D$ a \ref{for:vector_intersection_point_x} és \ref{for:vector_intersection_point_y} képletekben a nevező kiszámolásához biztosít könnyebb átláthatóságot, $(P_x, P_y)$ a kiszámolt metszéspont, $(x_1, y_1)$, $(x_2, y_2)$, $(x_3, y_3)$ és $(x_4, y_4)$ pedig a négy pont, amelyek a két egyenest határozzák meg, itt ezek közül az első kettő az egyik, a második kettő a másik egyeneshez tartozik.
\par A fenti egyenletek megvalósítása a \ref{cod:intersection} kódrészletben látható,

\vspace{2mm}
\hspace{-10mm}
\begin{minipage}{\linewidth}
\begin{lstlisting}[language=C++, numbers=left, caption={Metszéspont kereső algoritmus.}, label={cod:intersection}]
bool parallel = false;
cv::Point p1, p2, p3, p4, intersection;
float d, t1, t2;

d = (p1.x - p2.x) * (p3.y - p4.y) - (p1.y - p2.y) * (p3.x - p4.x);

if (std::abs(d) < 1e-8) {
    parallel = true;
}

if (!parallel) {
    t1 = (p1.x*p2.y - p1.y*p2.x) * (p3.x - p4.x) - (p3.x*p4.y - p3.y*p4.x) * (p1.x - p2.x);
    t2 = (p1.x*p2.y - p1.y*p2.x) * (p3.y - p4.y) - (p3.x*p4.y - p3.y*p4.x) * (p1.y - p2.y);
    
    intersection = cv::Point(t1 / d, t2 / d);
}
\end{lstlisting}
\end{minipage}

\par ahol \lstinline{p1}, \lstinline{p2}, \lstinline{p3}, \lstinline{p4} a fent megismert négy koordináta, \lstinline{d} a kiszámolt nevező, \lstinline{t1} és \lstinline{t2} pedig segédváltozók a számlálók tárolásához. A kódrészlet 7. - 9. soraiban látható, hogy abban az esetben ha \lstinline{d} nagyon kicsi, a metszéspont nem lesz kiszámolva. Ez azért van, mert a \ref{for:vector_intersection_denominator} függvényben kiszámolt nevező, $D = 0$ esetén a két egyenes párhuzamos, és ilyenkor nincs metszéspont.
\par A folyamat végeredményeképp kapott kép a \ref{fig:bemeneti_kep_quad} ábrán látható.

\begin{figure}[!ht]
    \centering
    \includegraphics[width=140mm, keepaspectratio]{figures/input_screen_quad.png}
    \caption{A felismert asztal négy pontból álló körvonala a bináris képen, piros körvonallal keretezve.}
    \label{fig:bemeneti_kep_quad}
\end{figure}

\section{Az asztal kivágása és torzítása}
Ahhoz, hogy az asztal a kontúr segítségével kivágható legyen a képből, szükség lesz egy téglalapra, amely alapján a kivágás elvégezhető. Ebben a részben ennek a folyamatnak a működéséről fogok beszélni.
\par A folyamat első részeként a kapott, négy koordinátából álló kontúr pontjait rendezni kell. A pontokat bal felső, jobb felső, jobb alsó és bal alsó pontok szerint kell sorba rendezni.
\newline Az átrendezéshez a \ref{cod:atrendezes} kódot használom,

\vspace{2mm}
\hspace{-10mm}
\begin{minipage}{\linewidth}
\begin{lstlisting}[language=C++, numbers=left, caption={Átrendező algoritmus.}, label={cod:atrendezes}]
std::vector<int> sums, diffs;

for (auto& point : quad) {
    sums.push_back(point.x + point.y);
    diffs.push_back(point.y - point.x);
}

std::vector<cv::Point2f> src;
src.push_back(quad[std::min_element(sums.begin(), sums.end()) - sums.begin()]);
src.push_back(quad[std::min_element(diffs.begin(), diffs.end()) - diffs.begin()]);
src.push_back(quad[std::max_element(sums.begin(), sums.end()) - sums.begin()]);
src.push_back(quad[std::max_element(diffs.begin(), diffs.end()) - diffs.begin()]);
\end{lstlisting}
\end{minipage}

\par ahol az előzőleg kiszámolt négy metszéspontot felhasználva, ahhoz hogy meg tudjam állapítani a azok relatív helyzetét, készítek az egyes pontokból összegeket (\lstinline{sums}) és különbségeket (\lstinline{diffs}), amelyek az egyes koordináták $x$ és $y$ összetevőinek összegeiből vagy különbségeiből állnak. Ezekből az összegek és különbségekből megállapítható a pontok helyzete, tehát például a bal felső koordinátát az összegek közül a legkisebb érték, a bal alsót a különbségek közül a legnagyobb érték határozza meg, és így a többi koordinátát is. A fent említett műveletek a kódrészlet 9. - 11. soraiban láthatóak.
\par Az előző művelet után a sorba rendezett koordináták meghatározzák a transzformációhoz szükséges mátrix kiszámításához a forrás (\lstinline{src}) értékeket. A transzformációhoz szükség van még a célértékekre is.
\newline A célértékek a \ref{cod:destination} kód soraival adhatóak meg,

\vspace{2mm}
\hspace{-10mm}
\begin{minipage}{\linewidth}
\begin{lstlisting}[language=C++, numbers=left, caption={A kimeneti értékek megadása.}, label={cod:destination}]
std::vector<cv::Point2f> dst;

int width = 1024 - 1;
int height = 512 - 1;

dst.push_back(cv::Point(0, 0));
dst.push_back(cv::Point(width, 0));
dst.push_back(cv::Point(width, height));
dst.push_back(cv::Point(0, height));
\end{lstlisting}
\end{minipage}

\par itt a cél kép méretei egy érték páros formájában szerepelnek a \lstinline{width} és \lstinline{height} változókban. Az értékekből egyet való levonás az indexelés végett szükséges. A szélesség és magasság értékekkel ezután meg lehet adni a célértékeket a transzformációs mátrix elkészítéséhez, ezek a \lstinline{dst} változóba kerülnek.
\par A transzformáció végrehajtásához mindezek után már csak a transzformációs mátrix elkészítésére van szükség, majd a transzformáció végrehajtására.
\newline Ezek a \ref{cod:pers_transform} kódrészlettel hajthatóak végre,

\vspace{2mm}
\hspace{-10mm}
\begin{minipage}{\linewidth}
\begin{lstlisting}[language=C++, numbers=left, caption={A transzformáció végrehajtása.}, label={cod:pers_transform}]
cv::Mat M = cv::getPerspectiveTransform(src, dst);
cv::warpPerspective(image, warp, M, cv::Size(width, height));
\end{lstlisting}
\end{minipage}

\par itt \lstinline{M} a transzformációs mátrix, amely a \lstinline{cv::getPerspectiveTransform} függvénnyel \cite{opencv_docs} kapható meg a forrás és célértékek megadásával. A függvény Gauss-elimináció \cite{grcar2011mathematicians} segítségével számol ki egy $3\times3$ méretű mátrixot, amelyet a \lstinline{cv::warpPerspective} függvénnyel \cite{opencv_docs} alkalmazok az \lstinline{image} változóban tárolt képre az \lstinline{M} mátrix és \lstinline{cv::Size(width, height)} méret megadásával. A transzformáló függvény lineáris interpolációt \cite{blu0401interpolation} használ alapértelmezett esetben az intenzitás értékek meghatározásához, a torzított kép a \lstinline{warp} változóban kerül tárolásra.
\par A kivágott és torzított kép a már megismert \ref{fig:bemeneti_asztal2} ábrán látható.

\section{Körkeresés}
A körök detektálásához az ún. Hough transzformációt (Hough Transformation) fogom használni, ez a H.K. Yuen, J. Princen, J. Illingworth és J. Kittler et. al. 1990 \cite{YUEN199071} szerint abban az esetben, ha egy kör a kövekező \ref{for:hough_transform} függvénnyel írható le,

\begin{equation}
    (x - a)^2 + (y - b)^2 = r^2
    \label{for:hough_transform}
\end{equation}

\par ahol $a$ és $b$ a kör középpontjának koordinátái és $r$ a sugár, akkor a körvonal élének egy tetszőleges $x_i$, $y_i$ pontja átalakításra kerül egy $a$, $b$, $r$ paraméterek által meghatározott térben elhelyezkedő egyenes kör alapú kúppá.\cite{hough_transform,YUEN199071} Amennyiben az adott pontok egy körvonalon helyezkednek el, a kúpok metszeni fogják egymást a kör $a$, $b$, $r$ pontjainak megfelelően.\cite{YUEN199071}
\par Az algoritmus lefutása után a metszéspontok megadják az egyes körök pozícióját, amelyeket könnyedén tárolni lehet egy listában.
\par A körkereső algoritmus a programkód formájában a \ref{cod:hough_circle} kódrészletben látható.

\vspace{2mm}
\hspace{-10mm}
\begin{minipage}{\linewidth}
\begin{lstlisting}[language=C++, numbers=left, caption={A körkereső algoritmus.}, label={cod:hough_circle}]
cv::cvtColor(image, gray, cv::COLOR_BGR2GRAY);

float minRadius = 5;
float maxRadius = 12;
float minDistance = 12;
int circleThreshold = 40;
int circlePerfectness = 10;

std::vector<cv::Vec3f> circles;
cv::HoughCircles(
    gray,
    circles,
    cv::HOUGH_GRADIENT,
    1,
    minDistance,
    circleThreshold,
    circlePerfectness,
    minRadius,
    maxRadius);
\end{lstlisting}
\end{minipage}

\par A kódrészletben a Hough transzformációt a \lstinline{cv::HoughCircles} függvény\cite{opencv_docs} végzi el, ennek első paramétere egy szürkeárnyalatos kép, amely a kivágott asztal konvertálásával kerül bele a \lstinline{gray} változóba, a konvertálás a már megismert \lstinline{cv::cvtColor} függvénnyel\cite{opencv_docs} megy végbe.
\par A második paraméter a megtalált körök listáját tartalmazza, a harmadik az előzőleg megismert Hough transzformációs módszert\cite{opencv_docs,hough_transform,YUEN199071} adja meg a \lstinline{cv::HOUGH_GRADIENT} kulcsszóval, továbbá a negyedik paraméter a folyamathoz felhasznált képet skálázza. A skálázas az eredeti kép felbontást \lstinline{1} értékkel nem változtatja, \lstinline{2} értékkel felére csökkenti azt, fordított arányosságnak megfelelően\cite{opencv_docs}. Ez a paraméter a folyamat lefutásának gyorsítását teszi lehetővé, azonban, akárcsak a jelenlegi esetben, kisebb körök detektálásához jobb az eredeti felbontás megtartása.
\par A \lstinline{minRadius} (minimális sugár), \lstinline{maxRadius} (maximális sugár) és \lstinline{minDistance} (minimális távolság) paraméterek megadják a keresett körök tulajdonságait, így növelhető az algoritmus teljesítménye és csökkenthető a duplikációk előfordulása.
\par A \lstinline{circleThreshold} paraméter az algoritmus kezdeti feldolgozó lépéseként végrehajtott Canny éldetektálás felső paraméterét adja meg, az alsó paraméter ennek felével lesz egyenlő. Ez a Canny éldetektálás a \ref{section:megv_asztal_kontur} alfejezetben megismert módszerrel megegyezően hajtódik végre. Végül pedig, a \lstinline{circlePerfectness} névvel ellátott paraméter a körkeresés pontosságát szabályozza, minél kisebb az érték, annál kisebb a pontosság, amely több hamisan felismert kört eredményezhet.
\par A folyamat lefutása után a körök pozíciója és mérete ismeretében jelölni tudjuk őket a kivágott képen. Ezt szemlélteti a \ref{chapter:program_tervezes} fejezet \ref{fig:talalt_korok} ábrája.

\section{A detektált körök osztályozása}

\subsection{Osztályozás mintaillesztéssel}
\label{subsection:mintaillesztes_osztalyozas}
A mintaillesztés ún. Kereszt Korrelációval (Normed Cross Correlation) megy végbe, ennek a működése a \ref{for:cross_correlation} képleten alapul\cite{kaehler2016learning, opencv_docs}.

\begin{equation}
    R(x, y) = \frac{\sum_{x',y'}(T(x',y') \cdot I(x + x', y + y'))}{\sqrt{\sum_{x',y'}T(x',y')^2 \cdot \sum_{x',y'}I(x + x',y + y')^2}}
    \label{for:cross_correlation}
\end{equation}

\par A képletben az $x$ és $y$ az eredeti képen vizsgált terület bal felső sarkát, $x'$ és $y'$ a minta képnek az adott képpontját, $T$ a minta képet és $I$ az eredeti képet jelöli. Ez a folyamat egy kernelhez hasonlóan végigpásztázza a képet, majd a kapott érték mátrixból eldönthető, hogy mely pontokon volt a legnagyobb egyezés a mintával.
\par Ebben az esetben viszont nem a teljes asztal képén vizsgálom a minta egyezését, hanem csak egy adott felismert körhöz tartozó metszeten. Ilyenkor a minta és a kép mérete megegyezik, ezért a kapott mártirxból a legnagyobb érték fogja reprezentálni az egyezőség mértékét. Ahhoz, hogy egy adott körnek meg lehessen állapítani a színét, az összes mintával hasonítani kell, majd a művelet végén a legnagyobb értékkel rendelkező illesztéshez tartozó minta színe fogja megadni a felismert kör színét. A \ref{fig:adatkeszlet} ábrán látható sárga golyóhoz hasonló adatra elvégzett mintaillesztés eredményeit a \ref{fig:mintaillesztes_eredmeny} ábra mutatja.

\begin{figure}[!ht]
    \centering
    \includegraphics[width=140mm, keepaspectratio]{figures/match_values.png}
    \caption{Egy sárga golyó kivágott képére elvégzett mintaillesztések eredményei.}
    \label{fig:mintaillesztes_eredmeny}
\end{figure}

\par Az ábrán látható, hogy a mintaillesztés nehezen különbözteti meg az egyes színeket, szinte minden színhez 65\% -nál nagyobb megegyezési értékeket ad. Ez jól szemlélteti, hogy miért nem kerül közvetlen felhasználásra az alkalmazásban, azomban a adatkészlet készítéséhez felhasználható.
\par A mintaillesztés az alkalmazásban a \ref{cod:template_match} kódrészlet formájában kerül felhasználásra.

\vspace{2mm}
\hspace{-10mm}
\begin{minipage}{\linewidth}
\begin{lstlisting}[language=C++, numbers=left, caption={A mintaillesztés menete.}, label={cod:template_match}]
if (hsvMode) {
    cv::Mat templateHSV;
    cv::cvtColor(templateBGR, templateHSV, cv::COLOR_BGR2HSV);
    cv::matchTemplate(cutImageHSV, templateHSV, result, cv::TM_CCORR_NORMED);
}
else {
    cv::matchTemplate(cutImageBGR, templateBGR, result, cv::TM_CCORR_NORMED);
}

double maxValue;
cv::minMaxLoc(result, NULL, &maxValue);
\end{lstlisting}
\end{minipage}

\par Itt a \lstinline{hsvMode} logikai változó megadja, hogy a mintaillesztést BGR vagy HSV színformátumban végezze el a kód. A HSV módban történő illesztésnél szimplán átkonvertálom a mintát HSV formátumra a már jól ismert \lstinline{cv::cvtColor} függvénnyel\cite{opencv_docs}, majd mind a két módban úgyanúgy végrehajtom a mintaillesztést a \lstinline{cv::matchTemplate} függvénnyel\cite{kaehler2016learning,opencv_docs}. A függvény a \ref{for:cross_correlation} egyenlet alapján számolja ki a minta egyezőségét egy adott pontban. Paraméterként meg kell adni neki a képet amelyre a mintát szeretnénk illeszteni (\lstinline{cutImageBGR} vagy \lstinline{cutImageHSV}), a mintát (\lstinline{templateHSV} vagy \lstinline{templateBGR}), a kimeneti változónkat, amely ebben az esetben a \lstinline{result} mátrix, továbbá meg kell adnunk a mintaillesztés algorimust, ezt a \lstinline{cv::TM_CCORR_NORMED} konstans teszi meg.
\par A kapott mátrixból az egyezőséget reprezentáló értéket a \lstinline{cv::minMaxLoc} függvény\cite{opencv_docs} adja meg, ez a függvény kiszámolja, hogy a bemeneti mátrixon (\lstinline{result}) hol található és mennyi a minimum és maximum érték. Az egyezőség megállapításához ezesetben csak a maximum értékre van szükség, ezt a függvény harmadik paramétere adja meg \lstinline{maxValue} néven.
\par A fenti folyamat egy körre írja le a minta illesztését, ezt ciklusban az alkalmazás végrehajtja minden potenciális goylót leíró körre.

\subsection{Osztályozás neurális hálózattal}
A neurális hálózattal való osztályozás hasonlóan megy végbe, minta az előző alfejezetben megismert mintaillesztes. Az osztályozáshoz szükség van egy betanított neurális hálózatra, amit be kell tölteni az alkalmazásba, majd továbbítani neki a kör kivágott képét. A kivágott képet átadás előtt viszont módosítani kell, mégpedig úgy, hogy megegyezzen a neurális háló kívánt bemenetével.
\par Az általam használt modell bemenete egy olyan képet vár, amelynek hat intenzitásértéke van, azonban egy átlagos kivágott kép csupán három ilyen értékkel rendelkezik, ezek a BGR által reprezentált kék (blue), zöld (green) és piros (red) intenzitások, a másik három szükséges intenzitást a kép HSV értékei adják meg, ezek az árnyalat (hue), telítettség (saturation) és érték (value).
\par A kép intenzitásait összefűzve, majd azt továbbítva a hálózatnak, a kapott érték egy normalizált értékek listája, ami megadja, hogy egyes színek mennyire jellemzik a bemeneti képet. A \ref{fig:mintaillesztes_eredmeny} ábrához hasonlóan látható egy sárga golyó neurális hálózatos osztályozásának eredménye a \ref{fig:neuralis_osztalyozas_eredmeny} ábrán. Az ábrán látható, hogy míg a sárga érték nem mutatható teljesen az ábra intervallumán, hisz megegyezése közel 100\% -os (99,46\%), addig a többi színhez tartozó megegyezési értékek 0.1\% alá esnek. Ez is mutatja, hogy a módszer rendkívül pontosan meg tudja jósolni egy potenciálisan felismert golyó színét.

\begin{figure}[!ht]
    \centering
    \includegraphics[width=140mm, keepaspectratio]{figures/neural_values.png}
    \caption{Egy sárga golyó kivágott képén neurális hálózattal történt becslés eredményei.}
    \label{fig:neuralis_osztalyozas_eredmeny}
\end{figure}

\par A felismerés végrehajtásához használt kód a \ref{cod:neural_predict} kódrészletben látható.

\vspace{2mm}
\hspace{-10mm}
\begin{minipage}{\linewidth}
\begin{lstlisting}[language=C++, numbers=left, caption={A neurális hálóval történő osztályozás menete.}, label={cod:neural_predict}]
std::vector<cv::Mat> channels = {cutImageBGR, cutImageHSV};
cv::Mat mixed;
cv::merge(channels, mixed);

fdeep::tensor input = fdeep::tensor_from_bytes(
    mixed.ptr(),
    static_cast<std::size_t>(mixed.rows),
    static_cast<std::size_t>(mixed.cols),
    static_cast<std::size_t>(mixed.channels()),
    0.0f,
    1.0f
);

std::vector<float> result = model.predict(input).to_vector();
\end{lstlisting}
\end{minipage}

\par A \ref{cod:neural_predict} kódrészletben elsősorban a kép intenzitás érékeit fűzzük össze, ezt a \lstinline{cv::merge} függvény\cite{opencv_docs} teszi meg, ami bemenetként kéri a két képet (BGR és HSV) egy tömb formájában, a kimenetet pedig a \lstinline{mixed} változóba teszi bele.
\par A hat színcsatornával rendelkező \lstinline{mixed} mátrixot ezekután egy ún. tensor típusra kell átalakítani, amelyet a neurális hálózat közvetlenül kezelni tud. Az átalakításhoz a \lstinline{fdeep::tensor_from_bytes} függvény\cite{frugally_deep2016} szükséges, amelynek paraméterként át kell adni a mátrixunk memóriacímének mutatóját (\lstinline{mixed.ptr()}), ez alapján tudja megállapítani a függvény az adatok helyzetét a memóriában, továbbá még át kell adni a mátrix szélességét (\lstinline{mixed.cols}), magasságát (\lstinline{mixed.rows}) és a színcsatornák számát (\lstinline{mixed.channels}). Az utolsó két paraméter megadja, hogy az intenzitás értékek 0 és 1 értékek közé legyenek skálázva. Ez egy 0 és 255 értékek között mozgó intenzitásnál, például 130, $\frac{130}{255} \approx 0.51$ értéket jelent.
\par A tensor-ra alakítás után a \lstinline{model.predict} függvénnyel\cite{frugally_deep2016} lehet a hálózat jóslását elvégezni, amely kimenete a \lstinline{result} változóba kerül. A kimenet egy lebegőpontos értékek tömbje lesz, ennek elemei adják meg, hogy egyes színek mennyire jellemzik az osztályozott képet. A legnagyobb tömb elemhez tartozó index jelöli a jósolt színt, hasonlóan az előző \ref{subsection:mintaillesztes_osztalyozas} alfejezetben megismert mintaillesztéshez.

\section{Játékmenet vizsgálati szempontok kiszámolása}
\chapter{Felhasználói dokumentáció}

\section{Az alkalmazás használata}

\section{Finomhangolási paraméterek}
%\include{content/introduction}
%\include{content/thesis-format}
%\include{content/latex-tools}
%\include{content/template-usage}
\chapter*{\osszefoglalas}
\addcontentsline{toc}{chapter}{\osszefoglalas}
Ez az amaz


% Köszönetnyilvánítás - opcionális
%~~~~~~~~~~~~~~~~~~~~~~~~~~~~~~~~~~~~~~~~~~~~~~~~~~~~~~~~~~~~~~~~~~~~~~~~~~~~~~~~~~~~~~
%\include{content/acknowledgement}


% Táblázatok listája - opcionális
%~~~~~~~~~~~~~~~~~~~~~~~~~~~~~~~~~~~~~~~~~~~~~~~~~~~~~~~~~~~~~~~~~~~~~~~~~~~~~~~~~~~~~~
%\clearpage\phantomsection
%\listoftables
%\addcontentsline{toc}{chapter}{\listtablename}


% Irodalomjegyzék
%~~~~~~~~~~~~~~~~~~~~~~~~~~~~~~~~~~~~~~~~~~~~~~~~~~~~~~~~~~~~~~~~~~~~~~~~~~~~~~~~~~~~~~
\clearpage\phantomsection
%\pagenumbering{gobble}
\addcontentsline{toc}{chapter}{\bibname}
\bibliography{bib/mybib}


% Függelékek
%~~~~~~~~~~~~~~~~~~~~~~~~~~~~~~~~~~~~~~~~~~~~~~~~~~~~~~~~~~~~~~~~~~~~~~~~~~~~~~~~~~~~~~
%----------------------------------------------------------------------------
\appendix
%----------------------------------------------------------------------------
\chapter*{\fuggelek}\addcontentsline{toc}{chapter}{\fuggelek}
\setcounter{chapter}{\appendixnumber}
%\setcounter{equation}{0} % a fofejezet-szamlalo az angol ABC 6. betuje (F) lesz
\numberwithin{equation}{section}
\numberwithin{figure}{section}
\numberwithin{lstlisting}{section}
%\numberwithin{tabular}{section}

\section{dataset5}
A betanításhoz használt adatkészlet, színenként külön mappákba helyezve, összesen nagyjából 12000 képfájlt tartalmaz.

\section{train2.py}
A neurális hálózat betanításához használt Python szkript.

\section{Cut.hpp}
A kivágott képek kezeléséhez létrehozott osztályt tartalmazó fájl.

\section{Section.hpp}
A szakaszok kezeléséhez létrehozott osztályt tartalmazó fájl.

\section{BallLabel.hpp}
A golyók színének enum értékeit tartalmazó fájl.

\section{Template.hpp}
A mintaillesztés során használt mintákhoz létrehozott osztály fájlja.

\section{Ball.hpp és Ball.cpp}
A golyókat és azok függvényeit tartalmazó osztály header fájlja és implementációja.

\section{Recognition.hpp és Recognition.cpp}
A felismerési folyamatok függvényeit tartalmazó forrásfájl header deklaráció és implementáció.

\section{TrackbarWindow.hpp}
A finomhangolási paraméterek megjelenítéséhez használt segédosztály fájlja.

\section{main.cpp}
A felismerő fő folyamatait egybefoglaló központi forráskód. A felismerés folyamatához használt többi programrész ebből a fájlból kerül meghívásra.

%\label{page:last}
\end{document}
