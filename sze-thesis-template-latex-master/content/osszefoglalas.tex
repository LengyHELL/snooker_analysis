\chapter*{\osszefoglalas}
\addcontentsline{toc}{chapter}{\osszefoglalas}
A dolgozat során elsősorban ismertetésre került a snooker játék, annak szabályai és részletei, ez egy alapot szolgáltatott a dolgozat többi részének megértéséhez, azután bemutatásra kerültek az alkalmazás megvalósításához használt könyvtárak, azok felhasználási módjai és tulajdonságai.
\par A könyvtárak bemutatása után felvázolásra kerültek a felismeréshez szükséges lépések, hogyan történik meg az asztal felismerése, azon a golyók megtalálása és azonosítása, továbbá az elemzési szempontok kiszámolása. Szóba kerültek különféle módszerek, azoknak mi az előnye és hátránya és, hogy miért kerültek felhasználásra a későbbiekben.
\par Az átfogó ismertetés után a neurális hálózat betanítása került bemutatásra, itt szóba került az adatkészlet betöltése, annak letisztítása és szétválasztása a betanítási folyamathoz, továbbá a neurális hálózat rétegeinek beállítása, és maga a neurális hálózat betanítása és a modell elmentése.
\par A hálózat betanítása után a felismerő alkalmazás megvalósítása került ismertetésre, itt részletesebben megvizsgálásra kerültek az átvezető részben ismertetett módszerek, továbbá a programkód részletek.
\par A dolgozatban elkészített felismerő alkalmazás nem feltétlenül közvetlen felhasználásra lett tervezve, a dolgozat kereteiben nem került lehetőség minden lehetséges esetben felmerülő speciális problémák megoldására, bőven van lehetőség mind az asztal felismerés és a golyó azonosítás terén javításokra, azomban a dolgozat tartalma betekintést enged különféle megoldások leírására, amelyek segíthetik, ösztönözhetik a témakörben érdeklőket, problémáik megoldásához és információt meríthetnek saját kutatásukhoz.