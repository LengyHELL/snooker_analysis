\pagenumbering{gobble} % roman numbering as default
\setcounter{page}{1}

\selecthungarian

%----------------------------------------------------------------------------
% Kivonat Magyarul 
%----------------------------------------------------------------------------
\chapter*{Kivonat}%\addcontentsline{toc}{chapter}{Kivonat}

A dolgozat célja egy snooker játék felülnézetes videójának felismerése, azon található golyók helyzetének és színének megállapítása és különböző szempontok vizsgálata a golyók helyzetének változása alapján. A dolgozat nem csak megoldást nyújt egy snooker asztal felismeréséhez, hanem részletezi a megvalósításhoz használt képfeldolgozási és gépi tanulási eszközöket és ezen eszközökhöz használt TensorFlow és OpenCV könyvtárak metódusait és azok használatát.
\par A dolgozat tartalma képfeldolgozás terén legfőképp a HSV maszkolás, kontúrok vizsgálata és kördetektálás témákat, míg gépi tanulási módszerek terén a konvolúciós neurális hálózatokat és azok betanításához szükséges adatkészlet feldolgozását érinti.


\vfill
\selectenglish


%----------------------------------------------------------------------------
% Abstract in English
%----------------------------------------------------------------------------
\chapter*{Abstract}%\addcontentsline{toc}{chapter}{Abstract}

The aim of this thesis is to identify the position of the balls in a top view video of a snooker game, to determine the position and color of the balls and to investigate different aspects of the game based on the change in the position of the snooker balls. The thesis not only provides a solution for the recognition of a snooker table, but also introduces image processing and machine learning tools used for the implementation of the application, and the methods and usage of the TensorFlow and OpenCV libraries used to implement these tools.
\par The content of the thesis is mainly related to HSV masking, contour analysis and circle detection in the field of image processing, while in the field of machine learning it is related to convolutional neural networks and the processing of the dataset required for their training.

\vfill
\selectthesislanguage

\newcounter{romanPage}
\setcounter{romanPage}{\value{page}}
\stepcounter{romanPage}