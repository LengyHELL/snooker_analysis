\chapter*{\bevezetes}
\addcontentsline{toc}{chapter}{\bevezetes}

A dokumentumban szereplő projekt a snooker billiárdjáték felismeréséről, és elemzéséről ad egy átfogó leírást. Ez a vidófelvétel használatával végrehajtott felismerés az asztalon elhelyezkedő különböző színű golyók pozíciójának meghatározásán és azok színének megállapításán alapszik.
\par A következő fejezetekben először ismertető rész mutatja be a snooker játék szabályait, annak érdekében hogy könnyebben érthető legyen az egyes folyamatok során használt nyelvezet és gondolkodásmenet, továbbá átfogóan szemléltetésre kerülnek a felismerés folyamatai, hogy egy adott képből milyen módon lehet megállapítani egyes golyók helyzetét. A golyók helyzetén kívül lesz szó a játékmenet elemzéséről, amely magában foglalja egyes golyók távolságának, útvonalának, sebességének és egyéb szempontoknak a kiszámolását és vizualizálását.
\par A felismerés megvalósításához különféle képfeldolgozási eszközök, neurális hálózat alapú kép osztályozás kerül felhasználásra, amelyeket \textbf{C++} és \textbf{Python} programozási nyelven valósítok meg főként \textbf{OpenCV} és \textbf{Tensorflow} könyvtárak eszközeinek használatával. A megvalósítási lépések dokumentálása során ismertetem az alkalmazás belső folyamatainak végbemenetelét kódrészletek és egyenletek formájában is. A megvalósítási fejezet során részletesebben szóba kerülnek a tervezés során bemutatott módszerek, hogy egyes módszereknek mi a hátránya a többivel szemben, és hogy miért kerültek felhasználásra az alkalmazás elkészítéséhez, végül bemutatok egy rövid ismertető dokumentációt az alkalmazás használatáról és funkcionalitásairól, hogy az alkalmazással hogyan lehetséges egy videófelvétel felismerésének elindítása, a felismerés paramétereinek hangolása és egyéb apróbb funkcionalitások használata.