\chapter{Felhasznált szoftverek}

\section{Az OpenCV képfeldolgozási könyvtár}
Az OpenCV egy főként \textbf{valós idejű képfeldolgozáshoz} használt programozási függvénykönyvtár. A könyvtár többféle programozási nyelvekhez készült implementációval létezik (pl.: C++, Python, Java stb.)\cite{opencv_library}, ezek közül ebben a projektben Python programozási nyelven keresztül fogom használni.
\par A könyvtárból használt függvények segítségével kerülnek megnyitásra a képek, továbbá a képeken való műveletek (pl.: szürkeárnyalatolás, élkeresés) is a könyvtár segítségével lesznek végrehajtva. A későbbiekben lesz szó a könyvtárból használt függvényekről, azok működéséről nagyobb részletességben.

\section{Tensorflow a neurális hálózatokhoz}
A Tensorflow az OpenCV -hez hasonlóan egy függvénykönyvtár, azzal a különbséggel, hogy a könyvtár a \textbf{neurális hálózatok elkészítését és betanítását} teszi lehetővé.\cite{tensorflow2015-whitepaper} A neurális hálózatok közül itt főként neurális hálózatokat (Neural Network) fogok használni, amelyek a képfeldolgozás, kép osztályozás területén teljesítenek kiemelkedően. A könyvtár eszközeiről szintén részletesebben beszélek majd a későbbi fejezetekben.