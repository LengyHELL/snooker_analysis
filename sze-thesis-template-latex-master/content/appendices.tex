%----------------------------------------------------------------------------
\appendix
%----------------------------------------------------------------------------
\chapter*{\fuggelek}\addcontentsline{toc}{chapter}{\fuggelek}
\setcounter{chapter}{\appendixnumber}
%\setcounter{equation}{0} % a fofejezet-szamlalo az angol ABC 6. betuje (F) lesz
\numberwithin{equation}{section}
\numberwithin{figure}{section}
\numberwithin{lstlisting}{section}
%\numberwithin{tabular}{section}

\section{dataset5}
A betanításhoz használt adatkészlet, színenként külön mappákba helyezve, összesen nagyjából 12000 képfájlt tartalmaz.

\section{train2.py}
A neurális hálózat betanításához használt Python szkript.

\section{Cut.hpp}
A kivágott képek kezeléséhez létrehozott osztályt tartalmazó fájl.

\section{Section.hpp}
A szakaszok kezeléséhez létrehozott osztályt tartalmazó fájl.

\section{BallLabel.hpp}
A golyók színének enum értékeit tartalmazó fájl.

\section{Template.hpp}
A mintaillesztés során használt mintákhoz létrehozott osztály fájlja.

\section{Ball.hpp és Ball.cpp}
A golyókat és azok függvényeit tartalmazó osztály header fájlja és implementációja.

\section{Recognition.hpp és Recognition.cpp}
A felismerési folyamatok függvényeit tartalmazó forrásfájl header deklaráció és implementáció.

\section{TrackbarWindow.hpp}
A finomhangolási paraméterek megjelenítéséhez használt segédosztály fájlja.

\section{main.cpp}
A felismerő fő folyamatait egybefoglaló központi forráskód. A felismerés folyamatához használt többi programrész ebből a fájlból kerül meghívásra.