%--------------------------------------------------------------------------------------
% Elnevezések
%--------------------------------------------------------------------------------------
\newcommand{\sze}{Széchenyi István Egyetem}
\newcommand{\kvik}{Gépészmérnöki, Informatikai és Villamosmérnöki Kar}
\newcommand{\szeaut}{Automatizálási Tanszék}
\newcommand{\szetat}{Távközlési Tanszék}
\newcommand{\szeint}{Informatika Tanszék}

\newcommand{\aut}{Automatizálási Szakirány}
\newcommand{\infokom}{Infokommunikáció Szakirány}
\newcommand{\merninf}{Mérnökinformatikus Szakirány}

\newcommand{\keszitette}{Készítette}
\newcommand{\konzulens}{Konzulens}

\newcommand{\szakdolgozat}{Szakdolgozat}
\newcommand{\diplomaterv}{Diplomaterv}
\newcommand{\dolgozat}{Dolgozat}

\newcommand{\villBSc}{Villamosmérnöki BSc}
\newcommand{\villMSc}{Villamosmérnöki MSc}
\newcommand{\minfBSc}{Mérnökinformatikus BSc}

\newcommand{\pelda}{Példa}
\newcommand{\definicio}{Definíció}
\newcommand{\tetel}{Tétel}

\newcommand{\bevezetes}{Bevezetés}
\newcommand{\koszonetnyilvanitas}{Köszönetnyilvánítás}
\newcommand{\fuggelek}{Függelék}

% Opcionálisan átnevezhető címek
%\addto\captionsmagyar{%
%\renewcommand{\listfigurename}{Saját ábrajegyzék cím}
%\renewcommand{\listtablename}{Saját táblázatjegyzék cím}
%\renewcommand{\bibname}{Saját irodalomjegyzék név}
%}


\newcommand{\szerzo}{\szerzoVezeteknev{} \szerzoKeresztnev}
\newcommand{\konzulensA}{\konzulensAMegszolitas\konzulensAVezeteknev{} \konzulensAKeresztnev}
\newcommand{\konzulensB}{\konzulensBMegszolitas\konzulensBVezeteknev{} \konzulensBKeresztnev}
\newcommand{\konzulensC}{\konzulensCMegszolitas\konzulensCVezeteknev{} \konzulensCKeresztnev}

\newcommand{\selectthesislanguage}{\selecthungarian}

\bibliographystyle{unsrtnat}

\def\lstlistingname{lista}

\newcommand{\appendixnumber}{6}  % a fofejezet-szamlalo az angol ABC 6. betuje (F) lesz
